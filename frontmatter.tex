%-------------------------------------------------------------------------------
% Cover page
%-------------------------------------------------------------------------------
%
\makecoverpage


%-------------------------------------------------------------------------------
% Dedication (comment if none)
%-------------------------------------------------------------------------------
%
\dedication{
%{\it It is true that Nature begins by reasoning and ends by experience. Nevertheless, \\we must begin with experiments and try through it to discover the reason.}\\[20pt]
%Leonardo da Vinci 
{If the scientist had an infinity of time at their disposal,\\it would be sufficient to say to them,\\``Look, and look \it{carefully}.''}\\[20pt]
Henri Poincar\'e
}


%-------------------------------------------------------------------------------
% Abstract (English)
%-------------------------------------------------------------------------------
%
% COMMENT: the abstract should be limited to 2000 characters and the keywords
% to 250 characters because otherwise they will not fit in the mailing sheet.
%
\begin{abstract}
Micron to millimetre sized droplets, precisely generated or sustained in controlled environment, have great potential in myriads of engineering applications
functioning as the basic element to assemble metamaterials, deliver drugs, host surfactant, reduce friction and damp turbulence.
The interaction of droplets from pairwise to collective levels is the most important factor in controlling these processes,
yet little is known about the detailed mechanisms in various nonideal conditions.
The present thesis combines a number of studies aiming to elucidate the physical principles of droplet interactions and suspension flow
using both high- and low-fidelity numerical simulations.

We first study flow-assisted droplet assembly in microfluidic channels, seeking to harness the droplet interactions to produce photonic bandgap materials.
A novel interface-correction level set/ghost fluid method (ICLS/GFM) is developed to directly simulate liquid droplets under depletion forces.
Comparing to previous methods, ICLS/GFM conserves the global mass of each fluid using a simple mass-correction scheme,
accurately computes the surface tension and depletion forces under the same framework,
and has subsequently been applied to investigate the droplet clustering observed in a microfluidic experiment.
Our simulations, supported by theoretical derivations, suggest that the observed fast self-assembly arises from a combination of 
strong depletion forces, confinement-mediated shear alignments of the droplets, and fine-tuned inflow conditions of the microchannel.
However, the interplay of these 3D effects negates a simple droplet interaction model of parametric dependence,
rendering the design of microfluidic chips for photonic crystal fabrications difficult in practice.

The next objective of the thesis is the implementation of a minimal discrete-element lubrication/contact dynamics model for simulation of dense particle suspensions.
The main ingredients of the model include
(i) a frame-invariant, short-range lubrication model for spherical particles, and
(ii) a soft-core, stick/slide frictional contact model activated when particles overlap.
Since contact interactions dominate at high particle concentrations,
we expect the methodology to be applicable for probing the jamming of non-spherical particles and the rheology of foams as well.

Finally, we include two miscellaneous studies concerning the slippage property of liquid-infused surfaces and droplets statistics in a homogeneous turbulent shear flow.
Overall, results of these simulations provide detailed flow visualisations and qualitative dependence of the target functional on various governing parameters,
facilitating experimental and theoretical investigations to design more robust drag-reducing surfaces and predict droplet distributions in emulsions.

%
\keywords{droplets, suspension, multiphase flow, microfluidics, soft matter, rheology, depletion force, level set, ghost fluid, discrete element.}
%
\end{abstract}


%-------------------------------------------------------------------------------
% Abstrakt (Swedish) 
%-------------------------------------------------------------------------------
%
% COMMENT: the abstract should be limited to 2000 characters and the keywords
% to 250 characters because otherwise they will not fit in the mailing sheet.
%
\begin{abstrakt}
Mikro- till millimeter stora droppar, exakt genererade eller hållna i en kontrollerad miljö, har stor potential i många olika tekniska tillämpningar. De representerar en grundläggande teknik vid uppbyggnad av metamaterial, transport av läkemedel i kroppen, som bärare av ytaktiva medel, vid minskning av friktion och dämpning av turbulens. Växelverkan mellan droppar från parvis till kollektivnivå är den viktigaste faktorn för att kontrollera dessa processer; ändå är lite känt om de detaljerade mekanismerna vid olika icke-ideala förhållanden. I denna avhandling kombineras ett antal studier som syftar till att belysa de fysikaliska principerna för dropp-växelverkningar och suspensionsflöden med numeriska simuleringar av högre och lägre noggrannhet.

Vi studerar först flödesassisterad droppmontering i mikrofluidkanaler och försöker utnyttja dropp-växelverkningar för att producera fotoniska bandgapmaterial. En ny interface-correction level set/ghost fluid method (ICLS/GFM) är utvecklad för att direkt simulera vätskedroppar under inverkan av utarmningskrafter. Jämfört med tidigare metoder bevarar ICLS/GFM den totala massan för varje fluid med hjälp av ett enkelt masskorrigeringsschema, och beräknar exakt ytspänningen och utarmningskrafterna under samma omständigheter. Detta tillämpas sedan för att undersöka droppklustring, något som observerats i mikrofluidiska experiment. Våra simuleringar, med stöd av teoretiska härledningar, antyder att den observerade snabba klustringen uppstår på grund av en kombination av starka utarmningskrafter och inneslutningsförmedlade skjuvkrafter på dropparna samt finjusterade inflödesförhållanden för mikrokanalen. Men samspelet mellan dessa 3D-effekter omöjliggör en enkel parameterberoende dropp-växelverkningsmodell vilket gör att utformningen av mikrofluidiska chips för fotonisk kristallfabrikation är svår i praktiken.

Nästa fokus i avhandlingen är implementeringen av en minimal diskretelementsmörjnings-/kontaktdynamikmodell för simulering av täta partikelsuspensioner. Två huvudingredienser i modellen är (i) en referensram-invariant smörjmodell med kort räckvidd för sfäriska partiklar, och (ii) en stick/slip friktionskontaktmodell med mjuk kärna som aktiveras när partiklar överlappar varandra. Eftersom kontakt-växelverkningar dominerar fysikaliskt vid höga partikelkoncentrationer, förväntar vi oss att metodologin också är tillämplig för att undersöka inklämning av icke-sfäriska partiklar och reologi för skum.

Slutligen inkluderar vi också två studier rörande glidningsegenskaper hos vätske-bemängda ytor och droppstatistik i ett homogent turbulent skjuvflöde. Sammantaget ger resultaten av dessa simuleringar detaljerade flödesvisualiseringar, kvalitativt beroende av målfunktionen på olika reglerande parametrar,
underlättar, experimentellt och teoretiskt, utformningen av mer robusta dragreducerande ytor samt förutsäger droppfördelningar i emulsioner.

\noindent
{\small (Redigerad av {\it Hanno Essén}.)}
%
\nyckelord{droppar, suspension, flerfasflöde, mikrofluidik, mjukt material, reologi, utarmningskraft, nivåuppsättning, spökvätska, diskret element.}
%
\end{abstrakt}


%-------------------------------------------------------------------------------
% Preface page
%-------------------------------------------------------------------------------
%
\begin{preface}
	This thesis summarizes a selection of studies on droplet interactions and suspension flow
        that may find applications in microfluidics, soft matter, and rheology.
        A brief introduction of the basic physical concepts and numerical methods is presented in Part I,
        followed by seven journal articles or preprints in Part II, see below.
        For consistency, all papers are formatted in the \texttt{jfm} style as the rest of the thesis.
        Their contents remain faithful to the original publications.
\end{preface}


%-------------------------------------------------------------------------------
% Division of work between authors
%-------------------------------------------------------------------------------
%
\begin{divisionofwork}
	The main thesis advisor is Professor Luca Brandt (LB).
	Associate Professor Outi Tammisola (OT) acts as the co-advisor.

        \bigskip

	\paperitem
                LB directed the research.
                Z. Ge (ZG) and J-Ch. Loiseau developed, implemented and validated the methods.
                OT oversaw the entire project.
		ZG wrote the paper with inputs from all other authors.

	\paperitem
                Following Paper 1, ZG continued the research, performed the simulation
                and wrote the paper with inputs from the rest of the authors.

	\paperitem
                A. Leshansky (AL) directed the research.
                I. Fouxon (IF) derived the theory.
		ZG performed the simulation.
                IF wrote the paper with inputs from the rest of the authors.

	\paperitem
                AL directed the research.
                IF constructed the theory.
		B. Rubinstein performed the multipole expansion simulation.
                ZG performed the Navier-Stokes simulation.
                IF wrote the paper with inputs from all other authors.

	\paperitem
                LB conceived the project.
                M.E. Rosti (MER) implemented the method and performed the simulation.
		MER and ZG performed the droplet statistics and scaling analyses.
                S.S. Jain visualised the flow.
                M.S. Dodd analysed the turbulent kinetic energy budget.
                MER wrote the paper with inputs from the rest of the authors.

	\paperitem
                G. Kreiss initiated the project.
                H. Holmgren implemented the code and performed the preliminary Stokes simulation.
                M. Kronbichler simulated the phase field equations.
		ZG performed the final Stokes simulation and 
                wrote the paper with inputs from the rest of the authors.

	\paperitem
                LB led the research.
		ZG implemented the method and wrote the report with input from LB.
               

\end{divisionofwork}


%-------------------------------------------------------------------------------
% Additional publications (comment if none)
%-------------------------------------------------------------------------------
%
%\begin{otherpublications}
%	The following papers, although related, are not included in this thesis.
%
%  \paperitem%
%    {Anakin Skywalker \& Master Obi-Wan Kenobi}% Authors
%    {3639}% Year
%    {The light sabre: an elegant weapon for a more civilized age}% Title
%    {Jedi Journal of Weapons}% Journal
%    {33}% Volume
%    {2}% Number
%    {pp. 55--60}% Pages
%
%\end{otherpublications}


%-------------------------------------------------------------------------------
% Conferences (comment if none)
%-------------------------------------------------------------------------------
%
\clearpage
\begin{conferences}
        In addition to journal publications, part of the work in this thesis were
        presented in the following conferences and workshops. The presenting author is underlined.

        \bigskip

  \conferenceitem%
    {\underline{Z. Ge}, L. Brandt}% Authors
    {June 2016}% Year
    {Simulation of the self-assembly of colloidal particles in a microchannel}% Title
    {7$^{th}$ Summer school of Complex Motions in Fluids}% Conference
    {Zenderen, Twente, Netherlands}% Location

  \conferenceitem%        
    {Z. Ge, \underline{O. Tammisola}, J.Ch. Loiseau, L. Brandt}% Authors
    {September 2016}% Year
    {Direct numerical simulation of the self-assembly of colloidal particles in a micro-channel}% Title
    {1$^{st}$ International Conference on Multiscale Applications of Surface Tension}% Conference
    {Brussel, Belgium}% Location

  \conferenceitem%
    {\underline{Z. Ge}, L. Brandt}% Authors
    {November 2016}% Year
    {Do self-assembly colloidal droplets behave like droplets?}% Title
    {69$^{th}$ Annual Meeting of the APS Division of Fluid Dynamics}% Conference
    {Portland, Oregon, USA}% Location

  \conferenceitem%
    {\underline{Z. Ge}, O. Tammisola, L. Brandt}% Authors
    {June 2017}% Year
    {A fast mass-preserving interface-correction level set/ghost fluid method for colloidal suspensions under depletion forces}% Title
    {3$^{rd}$ International Conference on Numerical Methods of Multiphase Flows (ICNMMF-III)}% Conference
    {Tokyo, Japan}% Location

  \conferenceitem%
    {\underline{Z. Ge}, H. Holmgren, M. Kronbichler, L. Brandt, G. Kreiss}% Authors
    {September 2018}% Year
    {Effective slip over partially filled microcavities and its possible failure}% Title
    {12$^{th}$ European Fluid Mechanics Conference}% Conference
    {Vienna, Austria}% Location

  \conferenceitem%
    {\underline{Z. Ge}, O. Tammisola, L. Brandt}% Authors
    {November 2018}% Year
    {Flow-assisted droplet assembly in a 3D microfluidic channel}% Title
    {71$^{st}$ Annual Meeting of the APS Division of Fluid Dynamics}% Conference
    {Atlanta, Georgia, USA}% Location

  \conferenceitem%
    {\underline{Z. Ge}, O. Tammisola, L. Brandt}% Authors
    {December 2018}% Year
    {Flow-assisted droplet assembly in a 3D microfluidic channel}% Title
    {Scattering and Dynamics of Flowing Soft Material}% Conference
    {Lund, Sweden}% Location

  \conferenceitem%
    {\underline{Z. Ge}, O. Tammisola, L. Brandt}% Authors
    {February 2019}% Year
    {Flow-assisted droplet assembly in a 3D microfluidic channel}% Title
    {Colloidal Science \& Metamaterials (CSM)}% Conference
    {Paris, France}% Location

\end{conferences}


%-------------------------------------------------------------------------------
% Table of contents
%-------------------------------------------------------------------------------
%
\tableofcontents
