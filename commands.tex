% Insert HERE any additional commands

\newcommand{\lb}[1]{\textcolor{red}{#1}}
\newcommand{\Ge}[1]{\textcolor{black}{#1}}
\newcommand{\xx}[1]{\textcolor{black}{#1}}

\newcommand\ie{\textit{i.e. }}
\newcommand\viz{\textit{viz. }}
\newcommand{\e}[1]{\ensuremath{\times 10^{#1}}}

\newcommand\todo{\textcolor{red}{to do} }


\newcommand{\secrefC}[1]{\mbox{Section \ref{#1}}}
\newcommand{\secref}[1]{\mbox{section \ref{#1}}}
\newcommand{\tabrefC}[1]{\mbox{Table \ref{#1}}}
\newcommand{\tabref}[1]{\mbox{table \ref{#1}}}
\newcommand{\equrefC}[1]{\mbox{Equation (\ref{#1})}}
\newcommand{\equref}[1]{\mbox{equation (\ref{#1})}}
\newcommand{\equrefM}[1]{\mbox{equations (\ref{#1})}}
\newcommand{\equrefMM}[1]{\mbox{(\ref{#1})}}
\newcommand{\figrefC}[2][]{\mbox{Figure \ref{#2}(#1)}}
\newcommand{\figref}[2][]{\mbox{figure \ref{#2}(#1)}}
\newcommand{\figrefSC}[1]{\mbox{Figure \ref{#1}}}
\newcommand{\figrefS}[1]{\mbox{figure \ref{#1}}}


% as examples here are some commands from the JFM template:
%
\newcommand\etal{\mbox{\textit{et al.}}}
\newcommand\etc{etc.\ }
\newcommand\eg{e.g.\ }

\providecommand\bnabla{\boldsymbol{\nabla}}
\providecommand\bcdot{\boldsymbol{\cdot}}

\newcommand\Rey{\mbox{\textit{Re}}}  % Reynolds number
\newcommand\Pran{\mbox{\textit{Pr}}} % Prandtl number, cf TeX's \Pr product
\newcommand\Pen{\mbox{\textit{Pe}}}  % Peclet number
