\begin{acknowledgements}
%  Advisors,
%  funders,
%  collaborators,
%  mentors,
%  colleagues,
%  friends,
%  family.

  The first part of this thesis was written from December 2019 to January 2020, during the warmest winter since I arrived in Stockholm five years ago.
  Looking back, I have encountered, experienced and learned so much from uncountable people that is nearly impossible to name them one by one.
  I am grateful to \emph{everyone} and apologize in advance for any negligence.

  First and foremost, I thank Professor Luca Brandt for accepting me in and guiding me through the PhD program at KTH Mechanics.
  His strong leadership from the inception,
  enthusiastic responses and constructive suggestions at every discussion,
  generous support and increasing trust in me towards the end are indispensable for me to grow and develop academically.
  For these reasons, I owe him a debt of gratitude.

  Second, I wish to extend my sincere thanks to my day-to-day advisors and mentors at KTH.
  I thank Outi Tammisolar for co-advising me since my first project, following my progress throughout the PhD
  and offering me invaluable career advice in the later years.
  I thank Jean-Christophe Loiseau for instilling in me the importance of proper programming and mathematical rigour.
  And I thank Mehdi Niazi for always being available for technical discussions and sharing his wisdom whenever I am lost.
  The patience, support and encouragement they offered are instrumental in building my confidence and independence, which I will never take for granted.

  Third, I want to express my appreciation to all collaborators and senior colleagues that have shaped and contributed to my PhD work.
  These include, but are not limited to, international collaborators: Michael Dodd, Alexander Leshansky, Itzhak Fouxon and Naoki Takeishi;
  former and present colleagues at KTH: Walter Fornari, Pedro Costa, Lailai Zhu, Francesco de Vita, Marco Rosti, Sagar Zade, B. M. Ningegowda, U\'{g}is L\={a}cis and Oleksiy Klurman;
  office mates and cultural ambassadors: Ricardo Vinuesa, Ekaterina Ezhova, Tímea Kékesi, Arash Alizad Banaei and Ashwin Vishnu;
  and professors that I enjoy talking to or playing innebandy with:
  Shervin Bagheri, Christophe Duwig, Hanno Essén, Lanie Gutierrez-Farewik, Ruoli Wang and Fredrik Lundell.
  Many others have also helped, corrected or inspired me in various research projects;
  my thanks are expressed in the separate acknowledgements after each paper in the next part of the thesis.
  
  In his essay \emph{\small The World As I See It}, Einstein wrote,
  ``But from the point of view of daily life, without going deeper, we exist for our fellowmen --
  in the first place for those on whose smiles and welfare all our happiness depends,
  and next for all those unknown to us personally with whose destinies we are bound up by the tie of sympathy.''
  In the last several years, I have been lucky to derive plentiful happiness and sympathy from Artem, Erik, Guillaume, Nicolas, Frida, Natasha, Thea, Lee,
  as well as the rest of the Roslagstull corridor.
  Particularly, I am blessed to meet Dia, whose courage, ambition, curiosity and sensitivity have become a continuing source of my attachment,
  making me feel thorough and complete.
  
  Finally, none of the above would have been possible without the enduring love and education from my family at home.
  
  \chinese{爸,妈}
  
  \chinese{感谢你们对我从小到大的培养}

  \chinese{感谢你们对我的一切付出}

  \chinese{我的所有都属于你们}
  
  
\end{acknowledgements}
