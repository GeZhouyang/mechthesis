%% teaser
\begin{frame}

  \frametitle{But what are \textit{droplets}?}

  \bigskip \bigskip
  \begin{columns}[T]
    
    \begin{column}{0.38\textwidth}
      \includegraphics[height=4.8cm]{imgs/droplet_martin-brechtl-unsplash.png}
      \vskip.3cm
      Droplets are micron to millemetre sized liquid balls formed under surface tension.%
      \footnote<.->[frame]{Photo by Martin Brechtl on Unsplash.}
    \end{column}

    \pause
    \begin{column}{0.45\textwidth}
      \movie[showcontrols]{\includegraphics[height=4.8cm]{imgs/St_Feynman.png}}
            {videos/droplet-Feynman.mp4}
      \vskip.3cm
      BBC interview of Richard Feynman (1983).%
      \footnote<.->[frame]{Source: \texttt{https://youtu.be/P1ww1IXRfTA}.}
      
      \pause
      \vskip.3cm
      %\centering
      \textcolor{bb}{\bf Droplets are fun subjects!}
    \end{column}
  \end{columns}

  
\end{frame}

%% outline
\begin{frame}{Outline}
  \protect\hypertarget{outline}{}
  \tableofcontents
\end{frame}

%%
\hypertarget{part1}{%
  \section{Part I: Fabricating Photonic Crystals (PhC)}}

%% 
\hypertarget{background1}{%
  \subsection{Background, motivation and challenge}}

\begin{frame}
  \frametitle{Background, motivation and challenge}

  \emph{Photonic crystals} (PhC) are materials patterned with a periodicity in dielectric constant
  and show great potential for building sophisticated optical circuitry that can
  route, filter, store or suppress optical signals.
  \bigskip
  
  \begin{columns}[T]
    
    \begin{column}{0.6\textwidth}
      \begin{figure}
        \centering 
        \includegraphics[height=4.8cm]{imgs/photonic_crystal_fibre.png} 
        \caption{A scanning electron micrograph of a solid-core photonic crystal fibre and its far-field optical pattern.%
          \footnote<.->[frame]{P. Russell, Science: 358-362. (2003)}
          \textcopyright \enspace AAAS Science.}
      \end{figure}
    \end{column}
    \hskip0.5cm
    \pause
    \begin{column}{0.3\textwidth}
      \begin{figure}
        \centering 
        \includegraphics[height=4.8cm]{../imgs/photonics_cover.jpg} 
        \caption{An artistic rendering of a light beam passing through a 3D photonic crystal.
          \textcopyright \enspace Springer Nature (2008).}
      \end{figure}
    \end{column}
  \end{columns}
  
\end{frame}

\begin{frame}
  \frametitle{Background, motivation and challenge}

  Despite the theoretical promise, fabricating photonic crystals with 3D bandgaps is challenging in practice.
  \bigskip
  
  \pause
  \begin{columns}[T]
    \begin{column}{0.45\textwidth}
      \begin{figure}
        \centering
        \includegraphics[height=4.8cm]{imgs/diamond-111.png}
        \caption{A diamond cubic (view angle 111), ideal structure of PhC.}
      \end{figure}
    \end{column}

    \pause
    \begin{column}{0.45\textwidth}
      \begin{figure}
        \centering
        \includegraphics[height=4.8cm]{imgs/patchy_colloid.png}
        \caption{Self-assembly of micron-sized patchy particles mimicking the arrangements of bonds around atoms.
        \textcopyright \enspace Springer Nature (2012).}
      \end{figure}
    \end{column}

  \end{columns}
  \vskip0.1cm
  
  \pause
  \centering{
    \xb{\bf The obtained clusters may serve as basic building blocks for fabricating PhC.}
    \vskip0.08cm
    \pause
    \xr{\bf The challenge is to speed up the process.}}

\end{frame}

%%
\hypertarget{experiment}{%
  \subsection{Experiments, strategy and questions}}

\begin{frame}
  \frametitle{Experiments, strategy and questions}

  A few years ago, a series of experiments%
  \footnote{B. Shen \etal Exp. Fluids, 55, 1. (2014); B. Shen \etal Adv. Sci., 3: 1600012. (2016).}
  performed at ESPCI, Paris suggested an alternative approach...
  \bigskip
  \begin{columns}[T]
    
    \begin{column}{0.52\textwidth}
      \begin{figure}
        \centering 
        \movie[showcontrols]{\includegraphics[height=3cm]{imgs/espci-delta.png}}{videos/espci-3drops.mp4}
        \movie[showcontrols]{\includegraphics[width=6.5cm]{imgs/espci-tetra.jpg}}{videos/espci-4drops.mp4}
        \caption{Assembly of two to four droplets in a microfluidic chip.}
      \end{figure}
    \end{column}

    \pause
    \begin{column}{0.4\textwidth}
      \begin{figure}
        \centering 
        \includegraphics[height=4.5cm]{imgs/espci-cluster-morph.png}
        \caption{Observed cluster morphologies.
        Images courtesy of Dr.\ Bingqing Shen.}
      \end{figure}
    \end{column}

  \end{columns}
  
  \pause
  \xb{\bf Here, the droplet assembly is driven by the flow. Hence, it is much faster.} \vskip0.1cm
  \pause
  \xr{\bf The questions are, why do they organize into such patterns and can we further optimize it?}
  
\end{frame}

%%
\hypertarget{dipolar}{%
  \subsection{A simple q2D dipolar model}}

\begin{frame}
  \frametitle{A simple q2D dipolar model}

  Noticing the strong geometric confinement of the microfluidic channel, physicists at ESPCI proposed the following \emph{dipolar model}.

  \begin{columns}[T]
    
    \begin{column}{0.45\textwidth}
      \centering
      \only<1-3,5,6>{\includegraphics[height=4cm]{imgs/espci-chip-cartoon.png}}
      \only<1-3>{\vskip.3cm \includegraphics[height=1.6cm]{imgs/espci-pancake.png}}
      \only<1-3>{\includegraphics[height=1.9cm]{imgs/espci-depletion.png}}
      \only<4>{\vskip.5cm \includegraphics[height=6cm]{../imgs/dipolar.pdf}}
      \only<5,6>{\includegraphics[height=2.8cm]{imgs/espci-dipolar-cartoon.png} \\}
      \only<6>{\xr{\bf Is this picture really correct?}}
    \end{column}

    \pause

    \begin{column}{0.45\textwidth}
      
      \begin{bluecolorbox}[Obersvations \& hypotheses]
        \begin{itemize}
        \item Droplets are in quasi-two-dimensional (q2D) space.
        \item Depletion force attracts nearby drops.
        \item Hydrodynamic interactions (HI) lead to the rearrangement dynamics.
        \end{itemize}
      \end{bluecolorbox}

      \vskip.3cm
      \pause
      \begin{bluecolorbox}[Equations of motion (due to HI)]
        \begin{equation} \notag
          \begin{aligned}
            & {\bm u}^\infty(x,y,z) \approx -\frac{z(H-z)}{2\mu} \nabla p(x,y), \\
            & \delta{\bm u}_{ij} = {\bm B}_{ij} {\bm F}_j, \quad {\bm U}_i = {\bm u}^\infty_i + \sum_{j \ne i} \delta{\bm u}_{ij},
          \end{aligned}
        \end{equation}
        where
        \begin{equation} \notag
          \begin{aligned}
            {\bm B}({\bm x}) \approx -\frac{\alpha H}{\mu} \bigg( \frac{{\bm I}}{\abs{\bm x}^2} -
            \frac{ 2{\bm x}{\bm x} }{ \abs{{\bm x}}^4} \bigg).
          \end{aligned}
        \end{equation}
      \end{bluecolorbox}
    \end{column}

  \end{columns}
    
\end{frame}

%%
\hypertarget{simulation1}{%
  \subsection{3D numerical simulations}}

\begin{frame}
  \frametitle{3D numerical simulations}

  To verify the model and subsequently optimize the chip design, we set up direct numerical simulations (DNS) of droplets in microchannels.
  \vskip0.3cm

  \pause
  \begin{columns}[T]

    \begin{column}{0.55\textwidth}
      \begin{bluecolorbox}[Mathematical formulation]
        The incompressible Navier-Stokes (NS) equations read,
        \begin{subequations} \label{eq:Navier-Sotkes}
          \begin{equation}
            \nabla \cdot {\bm u} = 0,
            \label{eq:div-free}
          \end{equation}
          \begin{equation}
            \rho \bigg(\frac{\partial {\bm u}}{\partial t} + {\bm u} \cdot \nabla {\bm u} \bigg) = \nabla \cdot {\bm \sigma} + {\bm f},
            \label{eq:NS}
          \end{equation}
        \end{subequations}
        where $\bm \sigma$ is the viscous stress tensor, given as
        \begin{equation}
          \begin{aligned}
            {\bm \sigma} = -p {\bm I}+ \mu \bigg( \nabla {\bm u} + (\nabla {\bm u})^T \bigg),
          \end{aligned}
        \end{equation}
        and is subject to the following boundary condition across the interface
        \begin{equation} \label{eq:stress-bc}
          ({\bm \sigma}_+ - {\bm \sigma}_- ) \cdot {\bm n} = \gamma \kappa {\bm n} - \nabla \gamma.
        \end{equation}
      \end{bluecolorbox}
    \end{column}

    \pause
    \begin{column}{0.4\textwidth}
      \centering
      \begin{bluecolorbox}[Non-dimensional numbers]
        \begin{equation} \notag
          \begin{aligned}
            \textrm{Re} = \frac{\rho UL}{\mu}, \quad
            \textrm{Ca} = \frac{\mu U}{\gamma}, \quad
            \textrm{Fr} = \frac{U^2}{FL}.  
          \end{aligned}
        \end{equation}
      \end{bluecolorbox}
      \vskip0.2cm

      \pause
      \begin{flushleft}
        In the experiments, %Re $\sim 10^{-2}$, Ca $\sim 10^{-5}$, Fr $\sim 10^{-5}$,
        \begin{equation} \notag
          \begin{aligned}
            \textrm{Re} \sim 10^{-2}, \quad
            \textrm{Ca} \sim 10^{-5}, \quad
            \textrm{Fr} \sim 10^{-5}.  
          \end{aligned}
        \end{equation}
        It means that inertial effects are negligible,
        the droplets will remain mostly spherical,
        and gravity matters.
        \vskip0.4cm

        \pause
        In developing the numerical solver, no such assumptions are made.
      \end{flushleft}
    \end{column}
    
  \end{columns}
  
\end{frame}

%%
\hypertarget{icls}{%
  \subsubsection{ICLS/GFM}}
%%1
\begin{frame}[t]
  \frametitle{Interface-resolved one-fluid methods}

  \begin{columns}[T]

    \begin{column}{0.45\textwidth}
      \vskip0.2cm
      \only<1-4>{
      There are numerous methods for direct simulation of droplets in fluids, such as
      \medskip
      \begin{itemize}
      \item the front-tracking methods
      \item the volume-of-fluid methods    
      \item the level set (LS) methods
      \item the phase field methods
      \item the lattice-Boltzmann methods
      \item the boundary integral methods
      \item ...
      \end{itemize}
      } % end only
      \medskip
      \only<2-4>{We use {\bf LS} for its simplicity and mathematical convenience.}

      \medskip
      \only<4>{However, there is an \xr{issue}...}
      \only<5->{\movie[showcontrols,poster,width=6cm,height=6.5cm]{}{videos/serpentine.mp4} \vskip0.2cm}
      \only<6->{\hskip1.2cm \xr{\bf LS is not mass-conserving.}}
      
    \end{column}

    
    \begin{column}{0.5\textwidth}
      \only<3->{
      \begin{bluecolorbox}[Classical LS formulation]

        The level set function, $\phi$, is classically defined as the signed distance to the interface $\Gamma$,
        \begin{equation} 
          \phi({\bm x},t) = sgn({\bm x}) |{\bm x}-{\bm x}_\Gamma|.
        \end{equation}
        For example, a sphere of radius r centered at origin can be represented as
        \begin{equation} \notag
          \Gamma = \{ {\bm x} ~ \rvert ~ \phi({\bm x},t) = 0 \},
        \end{equation}
        where
        \begin{equation} \notag
          \phi({\bm x},t) = |{\bm x}|^2 -r.
        \end{equation}
        The normal and the curvature of the interface can be readily calculated as
        \begin{equation}
          {\bm n} = \nabla \phi / \abs{ \nabla \phi }, \quad \kappa = \nabla \cdot {\bm n}.
        \end{equation}
        The interface motion is governed by
        \begin{equation} \label{eq:ls-avd}
          \frac{\partial \phi}{\partial t} + \bm{u} \cdot \nabla \phi = 0.
        \end{equation}
        
      \end{bluecolorbox}
      } %end only
    \end{column}
    
  \end{columns}

\end{frame}
%%2
\begin{frame}[t]
  \frametitle{Interface-correction level set/ghost fluid method (ICLS/GFM)}
  
  \begin{columns}[T]

    \begin{column}{0.45\textwidth}
      \vskip0.2cm
      \only<1->{
      To remedy the problem, we developed a simple mass-correction scheme.%
      \footnote<.->[frame]{Z. Ge \etal J. Comput. Phys. 353: 435-459. (2018)}
      } % end only
      \vskip0.2cm
      \only<2->{
      The key idea is to construct a correction-velocity, ${\bm u}_c$,
      such that the total volume is controlled by doing an additional advection.
      \vskip0.2cm} % end only
      \only<4>{\centering{\includegraphics[width=0.75\textwidth]{../paper1/Figures/mc.pdf}}} % end only
    \end{column}

    
    \begin{column}{0.5\textwidth}
      \only<3->{
      \begin{bluecolorbox}[ICLS formulation]

        The level set is defined and evolved in the same way as in the classical formulation,
        except that we also periodically examine the fluid mass loss.
        This allows us to define ${\bm u}_c$ as
        \begin{equation} \notag
          \int_\Gamma {\bm n} \cdot {\bm u}_c \,d\Gamma = \frac{\delta V}{\delta t},
        \end{equation}
        where $-\delta V/\delta t$ corresponds to the mass loss over an arbitrary period of time.
        If ${\bm u}_c$ is known, then solving
        \begin{equation}
          \frac{\partial \phi}{\partial t} + {\bm u}_c \cdot \nabla \phi = 0,
        \end{equation}
        will remove the mass loss accumulated over $\delta t$.

        It can be shown that
        \begin{equation}
          {\bm u}_c(\phi) = \frac{\delta V}{\delta t} \frac{\kappa(\phi)}{A_f} \delta_\epsilon (\phi),
        \end{equation}
        where $A_f = \int_\Gamma f_s \delta_\epsilon (\phi)|\nabla \phi| \,d\Gamma$.
        
      \end{bluecolorbox}
      } %end only
    \end{column}
    
  \end{columns}

\end{frame}
%%3
\begin{frame}[noframenumbering]
  \frametitle{Interface-correction level set/ghost fluid method (ICLS/GFM)}

  \centering
  \includegraphics[width=0.8\textwidth]{../paper1/Figures/serp_loss.pdf}
  \vskip0.3cm
  \xb{\bf A global mass conservation is now achieved.}

\end{frame}
%%4
\begin{frame}[noframenumbering]
  \frametitle{Interface-correction level set/ghost fluid method (ICLS/GFM)}

  \begin{columns}[T]
    
    \begin{column}{\textwidth}
      The flow solver also requires accurate computation of surface tension... \vskip0.2cm
      and a hydrodynamic model to impose the depletion force ... \vskip0.2cm
      \pause
      These details are given in: \vskip0.5cm
      \centering
      \begin{tcolorbox}[beamer,
          width=0.7\textwidth,
          arc=0pt,
          boxsep=1pt,
          left=0pt,right=0pt,top=0pt,bottom=0pt,
        ]
        \includegraphics[width=\linewidth]{imgs/JCP-cover.png}
      \end{tcolorbox}
    \end{column}
  \end{columns}

\end{frame}  
%%5
\begin{frame}[noframenumbering]
  \frametitle{Interface-correction level set/ghost fluid method (ICLS/GFM)}
  
  \begin{columns}
    
    \begin{column}{0.7\textwidth}
      \vskip0.3cm
      \only<1->{
      \begin{algorithm}[H] \notag
        \ccc{\footnotesize //time marching} \\
        \For{$n=1,2,\dots,N$}{
          
          \ccc{\footnotesize //level set advection} \\
          $\phi^1 = \phi^{(n)} + \Delta t \cdot {AD}(\phi^{(n)})$ \\
          $\phi^2 = \frac{3}{4} \phi^{(n)} + \frac{1}{4} \phi^1 + \frac{1}{4} \Delta t \cdot {AD}(\phi^1)$ \\
          $\phi^{(n+1)} = \frac{1}{3} \phi^{(n)} + \frac{2}{3} \phi^2 + \frac{2}{3} \Delta t \cdot {AD}(\phi^2)$ \\
              {\small Calculate $\rho^{(n+1)}$, $\mu^{(n+1)}$, $\bm{n}$, and $\kappa$ using $\phi^{(n+1)}$ } \\

              \ccc{\footnotesize //correct and reinitialize the level set every $N_i$ steps} \\
              \If{$n \mod N_i =0$}{
                {\small Mass-correction: advect with ${AD}(\phi) = (\delta V/\delta t) (\kappa \delta(\phi)/A_f)$} \\
                {\small Reinitialization: advect with ${AD}(\phi) = S(\phi)(\abs{\nabla \phi} -1)$} \\
              }
              \ccc{\footnotesize //flow solver (AB2)} \\
              ${\bf RU}^{(n)} = -{\bm u}^{(n)} \cdot \nabla {\bm u}^{(n)} +\frac{1}{Re}\big(\frac{1}{\rho^{(n+1)}} \nabla \cdot \big[\mu^{(n+1)}(\nabla {\bm u}^{(n)}+(\nabla {\bm u}^{(n)})^T)\big]\big)$ \\
              ${\bm u}^* ={\bm u}^{(n)} +\Delta t \big(\frac{3}{2} {\bf RU}^{(n)}-\frac{1}{2} {\bf RU}^{(n-1)} \big)$ \\
              $\nabla ^2 p^{(n+1)} = \nabla ^2_g [p]_\Gamma + \nabla \cdot \big[ \big(1-\frac{\rho_0}{\rho^{(n+1)}}) \nabla_g \hat{p} \big] + \frac{\rho_0}{\Delta t} \nabla \cdot {\bm u}^*$ \ccc{\footnotesize //FastP*-GFM} \\
              ${\bm u}^{(n+1)} = {\bm u}^* -\Delta t \big[\frac{1}{\rho_0} \nabla_g p^{(n+1)} + \big(\frac{1}{\rho^{(n+1)}} - \frac{1}{\rho_0}\big) \nabla_g \hat{p} \big]$ \ccc{\footnotesize //correction} \\
              ${\bf RU}^{(n-1)}={\bf RU}^{(n)}$ \\
        }
        \caption{A pseudo-code of ICLS/GFM.}
      \end{algorithm}
      }%end only
    \end{column}
    
    \begin{column}{0.2\textwidth}
      \only<2->{
        \vskip5.5cm
        \centering
        \includegraphics[width=1.2cm]{imgs/github.png}
        \vskip0.1cm
        Available at \texttt{ICLS-release}
      } %end only
    \end{column}
  \end{columns}

\end{frame}
%%5
\begin{frame}[noframenumbering]
  \frametitle{Interface-correction level set/ghost fluid method (ICLS/GFM)}

  \begin{columns}
    
    \begin{column}{0.9\textwidth}
      
      \begin{bluecolorbox}[Summary of numerics]
        \medskip
        \begin{itemize}
        \item The LS part is solved by standard techniques for hyperbolic partial differential equations. \medskip
        \item The NS part is solved by a standard projection method. \medskip
        \item The equations are solved on a staggered Cartesian grid using the finite volume method. \medskip
        \item The temporal integration is the 2nd-order Adam-Bashforth scheme. \medskip
        \item The pressure Poisson equation is solved using a fast Fourier transform based solver. \medskip
        \item The pressure jump across an liquid interface is imposed by the ghost fluid method (GFM). \medskip
        \item The depletion force between droplets is also computed under the GFM framework. \medskip
        \end{itemize}
      \end{bluecolorbox}
       
    \end{column}
    
  \end{columns}

  \vskip0.5cm
  \pause
  \centering
  \xb{\bf Now, we are ready to simulate the droplets.}
  
\end{frame}

%%%
%\hypertarget{ibm}{%
%  \subsubsection{NS/IBM}}


%%
\hypertarget{assembly}{%
  \subsection{Flow-assisted assembly}}

\begin{frame}[t]
  \frametitle{Flow-assisted assembly}

  \begin{columns}[T]
    
    \begin{column}{0.5\textwidth}      
      \only<1->{In our microfluidic device, the droplet dynamics are affected by:
      \vskip0.3cm
      \begin{itemize}
      \item near-field depletion force
      \item shear
      \item confinement
      \item boundary condition
      \end{itemize}
      \vskip0.3cm} %end only
      \only<2>{We examine these effects one-by-one as in the paper below.}
    \end{column}

    \begin{column}{0.45\textwidth}
      \only<1->{
      \centering
      \includegraphics[height=4cm]{imgs/espci-chip-cartoon.png}} %end only
    \end{column}
    
  \end{columns}
  
  \only<2>{
    \begin{tcolorbox}[beamer,
        width=.6\textwidth,
        arc=0pt,
        boxsep=1pt,
        left=0pt,right=0pt,top=0pt,bottom=0pt,
      ]
      \includegraphics[width=\linewidth]{imgs/soft-mat.png}
    \end{tcolorbox}
  } %end only
    
\end{frame}

%%1
\begin{frame}[t]
  \frametitle{Flow-assisted assembly: Approaching droplets in quiescent flows}

  \begin{columns}[T]
    \begin{column}{0.55\textwidth}
      Dimensional analysis suggests the following scaling.  \vskip0.2cm
      \begin{itemize}
      \item The strength of the attraction $\propto \Pi=p_{os}/p$.
      \item The approaching time $\propto T_\pi = r_s/(R\Pi)$.
        \only<3->{\item The physical time scale, $\tau_\pi = r_s \mu /(R p_{os})$, is $\sim 1$ ns.%
          \footnote<3->[frame]{Typical experimental values:
            $r_s = 1$ nm, ${\mu} = 10^{-3}$ kg/m-s, ${R} = 10$ $\mu$m, and ${p}_{os} = 100$ Pa.}
          }
        \only<4->{\\ Therefore, the approaching can be considered instantaneous.}
      \end{itemize}
    \end{column}

    \begin{column}{0.4\textwidth}
      \begin{figure}
        \includegraphics[width=\columnwidth]{../paper2/figs/depletion_flow.png}
      \end{figure}

    \end{column}
    
  \end{columns}

  \vskip0.4cm
  \only<2-3>{
    \begin{figure}
      \centering
      \includegraphics[width=0.75\textwidth]{../paper2/figs/min_dist4.pdf}
      \begin{picture}(0,0)
        \put(-210,48){\includegraphics[height=1.4cm]{../paper2/figs/2dp.png}}
        \put(-55, 48){\includegraphics[height=1.4cm]{../paper2/figs/3dp.png}}
      \end{picture}
    \end{figure}
  } %end only

  \only<5>{
    \begin{figure}
      \centering
      \includegraphics[width=0.6\textwidth]{../paper2/figs/packings.png}
    \end{figure}
  } %end only

  \only<6>{ \vskip0.6cm
    \begin{columns}    
      \begin{column}{0.7\textwidth}
        \begin{bluecolorbox}[Characteristics of depletion force]
          In summary, the depletion attraction
          \medskip
          \begin{itemize}
          \item is local (activated at nearly touching);
          \item acts in the radial direction; 
          \item has a fast reaction time. 
          \end{itemize}
          \medskip
          Consequently, the effect of depletion force on the cluster morphology is to preserve the initial positions in the absence of noise. 
        \end{bluecolorbox}
      \end{column}
    \end{columns}
  }
    
\end{frame}


%%2
\begin{frame}[t]
  \frametitle{Flow-assisted assembly: Sticky droplets in shear-driven channel flows}

  \begin{columns}[T]
    \begin{column}{0.55\textwidth}

      %\movie[]{}
      %      {videos/v3_simple_shear_cycle.ogv}
            
    \end{column}
  \end{columns}
  
    
\end{frame}

%%
\hypertarget{conclusion1}{%
  \subsection{Conclusions}}

%%
\hypertarget{part2}{%
  \section{Part II: Modelling Dense Suspensions (DS)}}

%%
\hypertarget{background2}{%
  \subsection{Soft matter and rheology}}

%%
\hypertarget{simulation2}{%
  \subsection{Numerical modelling}}

%%
\hypertarget{sd}{%
  \subsubsection{SD}}

%%
\hypertarget{hlgd}{%
  \subsubsection{HLGD}}

%%
\hypertarget{conclusion2}{%
  \subsection{Outlook}}

%%
\hypertarget{summary}{%
  \section{Summary}}

%%
\hypertarget{acknowledgements}{%
  \section{Acknowledgements}}










\iffalse

\begin{frame}
  \frametitle{}

  \begin{bluecolorbox}[Hypotheses for cascade directions]  
  \begin{itemize}
  \item
    In Batchelor we believe.
  \item
    Yes we still do.
  \end{itemize}  
  \end{bluecolorbox}  

\end{frame}

\fi

%\begin{frame}{Background}
%\begin{columns}[T]
%\begin{column}{0.5\textwidth}
%\textbf{Atmospheric energy
%spectrum}%\footnote<.->[frame]{\citet{NastromGage1985} © AMS}
%
%%\includegraphics[width=\textwidth,height=0.9\textheight]{../imgs/NastromGage.png}
%\end{column}
%
%\begin{column}{0.5\textwidth}
%\pause
%
%Two inertial ranges, separated by scales:
%
%~
%
%\begin{itemize}
%\tightlist
%\item
%  planetary / synoptic scales \(E(k) \sim k^{-3}\)
%  %\includegraphics[width=0.7\textwidth,height=0.28\textheight]{../imgs/synoptic.jpg}
%\end{itemize}
%
%\pause
%
%\begin{itemize}
%\tightlist
%\item
%  mesoscales \(E(k) \sim k^{-5/3}\)
%  %\includegraphics[width=0.7\textwidth,height=0.28\textheight]{../imgs/mesoscale.jpg}
%\end{itemize}
%\end{column}
%\end{columns}
%
%\pause
%
%\centering{\alert{How do we theorize the mechanism behind these two inertial ranges?}}
%\end{frame}
%%
%%\begin{frame}{Two-dimensional turbulence}
%%\protect\hypertarget{two-dimensional-turbulence}{}
%%Kraichnan's theory of 2D
%%turbulence\footnote<.->[frame]{\citet{Kraichnan1967}}
%%
%%\begin{itemize}
%%\item
%%  \textbf{Vorticity} and \textbf{enstrophy} conservation: a strong
%%  constraint on cascade
%%\item
%%  Dual cascade:
%%  \[E(k) \sim \epsilon^{2/3}k^{-5/3},\quad E(k) \sim \eta^{2/3}k^{-3}\]
%%\end{itemize}
%%
%%\pause
%%
%%\begin{itemize}
%%\item
%%  Directions of cascades:
%%
%%  \begin{itemize}
%%  \item
%%    \(k^{-5/3}\) range: constant energy
%%    flux\footnote<.->[frame]{Similar to scaling due to \citet{Kolmogorov1941}, \citet{obukhov1941distribution} and \citet{obukhov1941spectral}}
%%    \(\epsilon\), \textbf{inverse} cascade
%%  \item
%%    \(k^{-3}\) range: constant enstrophy flux \(\eta\), \textbf{forward}
%%    cascade
%%  \end{itemize}
%%\end{itemize}
%%
%%\pause
%%
%%\begin{itemize}
%%\tightlist
%%\item
%%  Spatial scales of inertial ranges: ``a
%%  paradox''\footnote<.->[frame]{\citet{Frisch}}
%%  \includegraphics[width=\textwidth,height=0.5\textheight]{../imgs/cascade_horiz.png}
%%\end{itemize}
%%\end{frame}
%%
%%\begin{frame}{Possible explanations of mesoscale spectrum}
%%\protect\hypertarget{possible-explanations-of-mesoscale-spectrum}{}
%%\begin{bluecolorbox}[Hypotheses for cascade
%%directions]\label{hypotheses-for-cascade-directions}
%%
%%\begin{itemize}
%%\item
%%  \citet{Gage:1979} \& \citet{Lilly:1983}: \textbf{inverse energy
%%  cascade} as in \citet{Kraichnan1967}
%%\item
%%  \citet{Dewan:1979}: \textbf{forward energy cascade} as in
%%  \citet{Kolmogorov1941}
%%\end{itemize}
%%
%%\end{bluecolorbox}
%%
%%\pause
%%
%%\begin{bluecolorbox}[Vertical resolutions]\label{vertical-resolutions}
%%
%%\begin{itemize}
%%\tightlist
%%\item
%%  \citet{Waite-Bartello:2004} and \citet{Lindborg2006} :
%%  \textbf{stratified turbulence}. \(l_v \sim u/N \approx 1 \text{km}\).
%%\item
%%  \citet{Callies-Buhler-Ferrari:2016} : \textbf{inertia gravity waves}.
%%  Frequency \(\omega \approx f\). i.e.~\(l_v \approx\) 100 metres.
%%\end{itemize}
%%
%%\end{bluecolorbox}
%%
%%\pause
%%
%%~
%%
%%\begin{columns}[T]
%%\begin{column}{0.4\textwidth}
%%DNS of stratified turbulence supports:
%%\footnote<.->[frame]{\citet{Lindborg2006} }
%%
%%\begin{itemize}
%%\item
%%  \(k^{-5/3}\) spectrum
%%\item
%%  \textbf{forward cascade}
%%\item
%%  \textbf{fine vertical resolution} requirement
%%\end{itemize}
%%\end{column}
%%
%%\begin{column}{0.48\textwidth}
%%\pause
%%
%%General circulation
%%models\footnote<.->[frame]{\citet{Augier-Lindborg:2013}} shows:
%%
%%\begin{itemize}
%%\item
%%  \(k^{-5/3}\) spectrum
%%\item
%%  \textbf{forward cascade} in mesoscales
%%\item
%%  with \textbf{coarse resolution}: 24 pressure levels along vertical
%%\end{itemize}
%%\end{column}
%%\end{columns}
%%
%%\pause
%%
%%\vspace{20pt}
%%
%%\centering{{\alert{Minimum number of vertical levels? Is it possible with a single level model?}}}
%%\end{frame}
%%
%%\begin{frame}{Quasi geostrophic equation}
%%\protect\hypertarget{quasi-geostrophic-equation}{}
%%Quasi-geostrophic equation\footnote<.->[frame]{\citet{Charney1971}}
%%conserves an approximate \emph{potential vorticity}: \[\Dt{q} = 0,\]
%%\[ q = \nabla^2 \psi + \frac{\alert<2>{f_0}^2}{\tilde \rho}
%%  \left(\frac{\tilde \rho}{\alert<3>{N^2}} \alert<3>{\p_z \psi} \right) + \alert<2>{\beta} y, \]
%%
%%\pause
%%
%%\begin{itemize}
%%\item
%%  Incorporates {\alert<2>{rotation}} and {\alert<3>{stratification}} in
%%  a 2D model
%%\item
%%  \onslide<4->{Bridging \textbf{ideal 2D turbulence} to
%%  \textbf{atmospheric turbulence}}
%%\item
%%  \onslide<5->{Valid for \textbf{strong rotation}, lengths scales
%%  \textbf{smaller than planetary} scales}
%%\item
%%  \onslide<6->{No ageostrophic motion, for example: \textbf{inertial
%%  gravity waves}}
%%\item
%%  \onslide<7->{Reproduces \(k^{-3}\) spectrum}
%%\end{itemize}
%%
%%~
%%
%%\onslide<7->{\centering{{\alert{What about the $k^{-5/3}$ mesoscale spectrum?}}}}
%%\end{frame}
%%
%%\hypertarget{shallow-water-equations}{%
%%\subsection{Shallow water equations}\label{shallow-water-equations}}
%%
%%\begin{frame}{Properties of shallow water equations}
%%\protect\hypertarget{properties-of-shallow-water-equations}{}
%%\begin{block}{Governing equations}
%%\protect\hypertarget{governing-equations}{}
%%\note{\begin{itemize}
%%\tightlist
%%\item
%%  Inviscid equations conserves in a periodic domain (no boundary fluxes)
%%\end{itemize}}
%%
%%\begin{columns}[T]
%%\begin{column}{0.5\textwidth}
%%\begin{align*}
%%  \partial_t \alert<2>{\mathbf{u}} &= - (\alert<2>{\mathbf{u}}.\nabla) \alert<2>{\mathbf{u}}
%%      - \alert<3>c^2 \nabla \alert<5>h - \alert<4>f\mathbf{e_z} \times \alert<2>{\mathbf{u}}, \\
%%  \partial_t \alert<5>h &= - \nabla \cdot (\alert<5>h \alert<2>{\mathbf{u}}).
%%  \end{align*}
%%
%%\begin{itemize}
%%\item
%%  where,
%%
%%  \begin{itemize}
%%  \tightlist
%%  \item
%%    \onslide<2->{\(\alert<2>{\mathbf{u}}=\) horizontal velocity vector,}
%%  \item
%%    \onslide<3->{\(\alert<3>{c} =\) wave speed,}
%%  \item
%%    \onslide<4->{\(\alert<4>f\) = Coriolis parameter,}
%%  \item
%%    \onslide<5->{\(\alert<5>{h} = 1 + \eta\), non-dimensional height of
%%    fluid\footnote<.->[frame]{\citet{vallis_atmospheric_2017}}.}
%%  \end{itemize}
%%\end{itemize}
%%\end{column}
%%
%%\begin{column}{0.5\textwidth}
%%\onslide<5->{\includegraphics[width=1\textwidth,height=0.4\textheight]{../imgs/swe_eta_h.png}}
%%\end{column}
%%\end{columns}
%%
%%\onslide<6->
%%
%%\begin{greencolorbox}[with \textbf{good
%%properties}]\label{with-good-properties}
%%
%%\begin{itemize}
%%\tightlist
%%\item
%%  Conserves \textbf{energy} \(E = E_K + E_P\) and the sum \(E_K + E_A\),
%%  where \(E_A =\) available potential energy (A.P.E.)
%%\item
%%  Conserves \textbf{potential vorticity}, \(Q = (\zeta + f)/h\)
%%\item
%%  Equipartition of \(E_K\) and \(E_A\) over a wave period
%%\end{itemize}
%%
%%\end{greencolorbox}
%%
%%\onslide<7->
%%
%%\begin{redcolorbox}[and some
%%\textbf{downsides}]\label{and-some-downsides}
%%
%%\begin{itemize}
%%\tightlist
%%\item
%%  Waves \(\to\) shocks
%%\item
%%  Cubic \(E_K = h\mathbf{u.u} / 2\)
%%\end{itemize}
%%
%%\end{redcolorbox}
%%\end{block}
%%\end{frame}
%%
%%\begin{frame}{Results: Energy
%%cascade\footnote<.->[frame]{ \citet{augier_shallow_2019} }}
%%\protect\hypertarget{results-energy-cascade}{}
%%\begin{bluecolorbox}[SW analogue of \citet{Kolmogorov1941}'s
%%\(\frac{4}{5}\) law for \(3^{rd}\)-order structure
%%function]\label{sw-analogue-of-kolmogorov1941s-frac45-law-for-3rd-order-structure-function}
%%
%%\begin{equation*}
%%  \meane{ |\delta \uu|^2 \delta J_L }
%%  + c^2\meane{ (\delta h)^2 \delta u_L } = -4 \epsilon r,
%%  \end{equation*}
%%
%%\begin{itemize}
%%\tightlist
%%\item
%%  \(\epsilon =\) energy flux or dissipation; \(r =\) separation distance
%%\item
%%  \(J_L \equiv h u_L\) and \(u_L \equiv \uu\cdot\rr / |\rr|\) are
%%  longitudinal momentum and velocities
%%\item
%%  positive flux (\(\epsilon > 0\)) \(\Rightarrow\) \textbf{forward
%%  energy} cascade
%%\end{itemize}
%%
%%\end{bluecolorbox}
%%
%%\pause
%%
%%\begin{figure}
%%\centering
%%\includegraphics[width=\textwidth,height=0.5\textheight]{../paper_04_shallow_water/Pyfig/fig3-eps-converted-to.pdf}
%%\caption{Spectral energy fluxes \(\Pi(k)\) and \(3^{rd}\)-order
%%structure functions \(\approx-4\epsilon r\)}
%%\end{figure}
%%\end{frame}
%%
%%\begin{frame}{Results: Shock
%%waves\footnote<.->[frame]{ \citet{augier_shallow_2019} }}
%%\protect\hypertarget{results-shock-waves}{}
%%\begin{columns}[T]
%%\begin{column}{0.65\textwidth}
%%\begin{figure}
%%\centering
%%\includegraphics[width=\textwidth,height=0.8\textheight]{../paper_04_shallow_water/Pyfig/fig5-eps-converted-to.pdf}
%%\caption{Visualization of shocks using divergence \(\nabla.\uu\) and
%%velocity component \(u_y\)}
%%\end{figure}
%%\end{column}
%%
%%\begin{column}{0.48\textwidth}
%%\begin{block}{Highlights}
%%\protect\hypertarget{highlights}{}
%%\begin{itemize}
%%\item
%%  Parameters
%%
%%  \begin{itemize}
%%  \tightlist
%%  \item
%%    Top: \(n=1920\), \(c=20\)
%%  \item
%%    Bottom: \(n=1920\), \(c=400\)
%%  \end{itemize}
%%\item
%%  \(F_f \propto 1 / c\)
%%\item
%%  \(\nabla. \uu < 0 \Rightarrow\) shock
%%\end{itemize}
%%\end{block}
%%\end{column}
%%\end{columns}
%%\end{frame}
%%
%%\begin{frame}{Results: Spectra and higher-order
%%statistics\footnote<.->[frame]{ \citet{augier_shallow_2019} }}
%%\protect\hypertarget{results-spectra-and-higher-order-statistics}{}
%%\begin{bluecolorbox}[Scaling laws for shock dominated
%%turbulence]\label{scaling-laws-for-shock-dominated-turbulence}
%%
%%\begin{itemize}
%%\item
%%  \onslide<+->{Shock amplitudes,
%%  \(| \Delta u | \sim | c \Delta h | \sim (\epsilon d)^{1/3}\)}
%%\item
%%  \onslide<+->{\(p^{th}\)-order structure functions
%%  \[\meane{|\delta u |^p}  \sim \meane{(c|\delta h |)^p} \sim  (\epsilon
%%  d)^{p/3} \,  \frac{r}{d}\]}
%%\item
%%  \onslide<+->{\(p = 2 \Rightarrow\) energy spectra:
%%  \(E_K(k) \sim E_A(k) \sim \epsilon ^{2/3} d^{-1/3} k^{-2}\)}
%%\end{itemize}
%%
%%\end{bluecolorbox}
%%
%%\pause
%%
%%\vspace{-10pt}
%%
%%\begin{figure}
%%\centering
%%
%%\subfloat[\(d/L_f \sim F_f^{1/2}\)]{\includegraphics[width=\textwidth,height=0.44\textheight]{../paper_04_shallow_water/Pyfig/fig6.eps}\label{fig:D}}
%%\subfloat[\(E(k) \sim k ^{-2}\)]{\includegraphics[width=\textwidth,height=0.44\textheight]{../paper_04_shallow_water/Pyfig/fig10.eps}\label{fig:E}}
%%
%%\caption{Shock separation \(d\) and energy spectra \(E(k)\) scaling}
%%
%%\label{fig:scaling}
%%
%%\end{figure}
%%\end{frame}
%%
%%\hypertarget{toy-model-equations}{%
%%\subsection{Toy model equations}\label{toy-model-equations}}
%%
%%\begin{frame}{Derivation of toy model
%%equations\footnote<.->[frame]{\citet{LindborgMohanan2017}}}
%%\protect\hypertarget{derivation-of-toy-model-equations}{}
%%\note{with \(\Psi\) and \(\chi\) being the \textbf{stream function} and
%%the \textbf{velocity potential} respectively.}
%%
%%\begin{block}{Helmholtz decomposition}
%%\protect\hypertarget{helmholtz-decomposition}{}
%%\[{\bf u} = \bf{u}_r + \bf{u}_d\]
%%
%%\begin{itemize}
%%\tightlist
%%\item
%%  \({\bf u}_r = -\nabla \times ( {\bf e_z} \Psi)\) is the rotational
%%  component
%%\item
%%  \(\bf {u}_d = \nabla \chi\) is the divergent component
%%\end{itemize}
%%
%%\pause
%%\end{block}
%%
%%\begin{block}{Governing equations}
%%\protect\hypertarget{governing-equations-1}{}
%%\begin{itemize}
%%\tightlist
%%\item
%%  Starting from classical shallow water equations,
%%\end{itemize}
%%
%%\begin{bluecolorbox}[Assumptions \&
%%modifications]\label{assumptions-modifications}
%%
%%\begin{itemize}
%%\item
%%  \onslide<2->{{\alert<2>{\textbf{Surface displacement much smaller}}}
%%  compared to the mean fluid layer height, \(\eta << 1\).}
%%\item
%%  \onslide<4->{Velocities in the large scale are
%%  {\alert<4>{\textbf{dominated by rotational part}}},
%%  \(|\bf u_r| >> |\bf u_d|\).}
%%\item
%%  \onslide<6->{{\alert<6>{\textbf{Substitute}}} \(c\eta\) with
%%  \(\theta\) (optional).}
%%\end{itemize}
%%
%%\end{bluecolorbox}
%%
%%\begin{align*}
%%\frac{\partial {\bf u}} {\partial t} + {
%%  \color<4>{purple}\only<-4>{{\uu}\, \cdot\, \nabla}
%%  \color<5->{teal}\only<5->{{\uu_r}\cdot \nabla}
%%  }
%%  {\bf u} +
%%  f {\bf e}_z \times {\bf u} &=
%%  \color<6>{purple}\only<-6>{-c^2 \nabla \eta} %
%%  \color<7->{teal}\only<7->{-c \nabla \theta} %
%%  \\
%%\frac{\partial 
%%  \color<6>{purple}\only<-6>\eta
%%  \color<7->{teal}\only<7->\theta
%%}{\partial t}+ {
%%  \color<4>{purple}\only<-4>{{\uu}\, \cdot\, \nabla}
%%  \color<5->{teal}\only<5->{{\uu_r}\cdot \nabla}
%%  }
%%  \color<6>{purple}\only<-6>\eta
%%  \color<7->{teal}\only<7->\theta
%%  &= -
%%  \color<7->{teal}\only<7->c
%%  \color<2>{purple}\only<2>{(1+\eta) \nabla \cdot {\bf u}}
%%  \color<3->{teal}\only<3->{\nabla \cdot {\bf u}}
%%\end{align*}
%%
%%\begin{itemize}
%%\tightlist
%%\item
%%  \onslide<7->{\color{teal}{Q.E.D.}}
%%\end{itemize}
%%\end{block}
%%\end{frame}
%%
%%\begin{frame}{A good
%%compromise\footnote<.->[frame]{\citet{LindborgMohanan2017}}}
%%\protect\hypertarget{a-good-compromise}{}
%%\begin{greencolorbox}[Pros]\label{pros}
%%
%%\begin{itemize}
%%\item
%%  \textbf{No shocks}
%%\item
%%  Kinetic and available potential energies are \textbf{quadratic} and
%%  conserved
%%\item
%%  \textbf{Linearised potential vorticity} \(q\) conserved in the limit
%%  of strong rotation \(Ro \to 0\), where \(q=\zeta - f \eta\)
%%\end{itemize}
%%
%%\end{greencolorbox}
%%
%%\begin{redcolorbox}[Cons]\label{cons}
%%
%%\begin{itemize}
%%\tightlist
%%\item
%%  \textbf{Full potential vorticity} \(Q\) is not exactly conserved
%%\end{itemize}
%%
%%\end{redcolorbox}
%%
%%\begin{columns}[T]
%%\begin{column}{0.58\textwidth}
%%\centering{%
%%  {%
%%  \movie[
%%    width=8.5cm,
%%    height=4.3cm,
%%    showcontrols,
%%    poster,
%%    autostart, loop
%%  ]{}{./videos/toy_model_qa.mp4}
%%
%%  Video: Time lapse of potential vorticity ($q$) & wave field}
%%}
%%\end{column}
%%
%%\begin{column}{0.48\textwidth}
%%\begin{figure}
%%\centering
%%\includegraphics[width=\textwidth,height=4.3cm]{../paper_03_toy_model/fig13.eps}
%%\caption{Blown-up view of a strong anticyclonic vortex \& the wave
%%field}
%%\end{figure}
%%\end{column}
%%\end{columns}
%%\end{frame}
%%
%%\begin{frame}{Energy spectra}
%%\protect\hypertarget{energy-spectra}{}
%%\begin{columns}[T]
%%\begin{column}{0.65\textwidth}
%%\begin{figure}
%%\centering
%%
%%\subfloat[\(k_f = 6\delta k\)]{\includegraphics[width=\textwidth,height=0.75\textheight]{../paper_03_toy_model/fig3.eps}\label{fig:speck6}}
%%\subfloat[\(k_f = 30\delta k\)]{\includegraphics[width=\textwidth,height=0.75\textheight]{../paper_03_toy_model/fig11.eps}\label{fig:speck30}}
%%
%%\caption{Time averaged energy spectra from two simulations by forcing at
%%different forcing wavenumbers (\(k_f\))}
%%
%%\label{fig:spectratoy}
%%
%%\end{figure}
%%\end{column}
%%
%%\begin{column}{0.48\textwidth}
%%\begin{block}{Highlights}
%%\protect\hypertarget{highlights-1}{}
%%\begin{itemize}
%%\tightlist
%%\item
%%  Legend
%%
%%  \begin{itemize}
%%  \tightlist
%%  \item
%%    \(E_K =\) {\color{red}{kinetic energy}}
%%  \item
%%    \(E_A =\) {\color{blue}{available potential energy}}
%%  \item
%%    \(E_W =\) {\color{violet}{wave energy}}
%%  \item
%%    \(E_V =\) {\color{teal}{vortical energy}}
%%  \end{itemize}
%%\item
%%  \(E_W \sim k^{-5/3}\)
%%\item
%%  \(E_V \sim k^{-3}\)
%%\item
%%  Equipartition between \(E_K\) and \(E_A\)
%%\end{itemize}
%%\end{block}
%%\end{column}
%%\end{columns}
%%\end{frame}
%%
%%\begin{frame}
%%\note{The total spectral energy flux \(\Pi\) has been decomposed into
%%kinetic (\(\Pi_K\)) and available potential energy (\(\Pi_A\)) energy
%%fluxes. The conversion from available potential energy to kinetic energy
%%is represented by \(C_{cum}\). The kinetic energy flux is further
%%decomposed as \(\Pi_{2D}\), the flux due to geostrophic modes and the
%%difference \(\Pi_K - \Pi_{2D}\).}
%%
%%\begin{columns}[T]
%%\begin{column}{0.55\textwidth}
%%\begin{block}{Spectral energy budget}
%%\protect\hypertarget{spectral-energy-budget}{}
%%\begin{itemize}
%%\tightlist
%%\item
%%  \textbf{Spectral energy flux},
%%  \(\Pi(\mathbf{k},t) = \int_\mathbf{k}^{\infty} T(\mathbf{k}',t) d\mathbf{k}'\)
%%\end{itemize}
%%
%%\pause
%%
%%\begin{itemize}
%%\tightlist
%%\item
%%  \(T=T_K + T_A\) are \textbf{transfer functions}, \begin{align*}
%%  \partial_t E_K(\mathbf{k},t) 
%%    &= \text{\textonehalf} \partial_t (\hat{\uu}\hat{\uu}^*)
%%    = T_K + C_K
%%    ,\quad\\
%%  \partial_t E_A(\mathbf{k},t) 
%%    &= \text{\textonehalf} \partial_t (\hat{\theta}\hat {\theta}^*)
%%    = T_A + C_A 
%%  \end{align*} which for the toy model equations are derived as
%%  \(T_K= \Im\left[\hat{u}_i^* k_j \widehat{u^r_j u_i}\right],\quad T_A = \Im\left[\hat{\theta}^* k_j \widehat{u^r_j \theta}\right]\).
%%\end{itemize}
%%\end{block}
%%\end{column}
%%
%%\begin{column}{0.48\textwidth}
%%\pause
%%
%%\begin{block}{Normal mode decomposition}
%%\protect\hypertarget{normal-mode-decomposition}{}
%%\begin{itemize}
%%\tightlist
%%\item
%%  Linearization and diagonalization of the toy model equations yields
%%  the \textbf{normal modes}: \begin{align*}
%%      \mathbf{B}
%%      = & \frac{1}{\sqrt{2}\sigma}
%%      \begin{Bmatrix} \sqrt{2} c\left(-
%%          \kappa^{2} \hat \psi +  f\hat\eta \right)               \\
%%          \kappa \left(c^{2} \eta + f \hat{\psi} + i \hat{\chi} \sigma\right) \\
%%          \kappa \left(c^{2} \eta + f \hat{\psi} - i \hat{\chi} \sigma\right)
%%      \end{Bmatrix}
%%      \propto
%%      \begin{Bmatrix}
%%          q \\
%%          a^+ \\
%%          a^-
%%      \end{Bmatrix}
%%  \end{align*} which can be transformed to
%%  \(\mathbf{U} = \{\hat{u}_x,\hat{u}_y, \hat{\theta}\}^T\).
%%\end{itemize}
%%\end{block}
%%\end{column}
%%\end{columns}
%%
%%\pause
%%
%%\begin{columns}[T]
%%\begin{column}{0.75\textwidth}
%%\begin{figure}
%%\centering
%%\includegraphics[width=\textwidth,height=0.52\textheight]{../paper_03_toy_model/fig5.eps}
%%\caption{Spectral energy fluxes from two runs with different forcing
%%schemes}
%%\end{figure}
%%\end{column}
%%
%%\begin{column}{0.48\textwidth}
%%\vspace{40pt}
%%
%%\begin{block}{Highlights}
%%\protect\hypertarget{highlights-2}{}
%%\begin{itemize}
%%\item
%%  Forcing scheme differences:
%%
%%  \begin{itemize}
%%  \tightlist
%%  \item
%%    Left: waves
%%  \item
%%    Right: vortices \& wave
%%  \end{itemize}
%%\item
%%  \(V:\) vortical mode
%%\item
%%  \(W:\) wave mode
%%\end{itemize}
%%\end{block}
%%\end{column}
%%\end{columns}
%%\end{frame}
%%
%%\begin{frame}{Comparison of a
%%GCM\footnote<.->[frame]{\citet{Augier-Lindborg:2013}} with the toy
%%model\footnote<.->[frame]{\citet{LindborgMohanan2017}}}
%%\protect\hypertarget{comparison-of-a-gcm-with-the-toy-model}{}
%%\begin{figure}
%%\centering
%%
%%\subfloat[]{\includegraphics[width=\textwidth,height=0.5\textheight]{../paper_03_toy_model/fig1.eps}\label{fig:sebgcm}}
%%\subfloat[]{\includegraphics[width=\textwidth,height=0.5\textheight]{../paper_03_toy_model/fig10.eps}\label{fig:sebtoy}}
%%
%%\caption{Spectral energy budgets from (a) GCM and (b) toy model
%%simulations.}
%%
%%\label{fig:sebgcmtoy}
%%
%%\end{figure}
%%
%%\begin{itemize}
%%\tightlist
%%\item
%%  where, \(\Pi_{2D} \equiv \Pi_{VVV}\); \(C_{cum}=\) cumulative
%%  conversion function.
%%\item
%%  large-scale forcing in A.P.E. \(\to\) baroclinic instability \(\to\)
%%  conversion to K.E. \(\to\) mesoscale turbulence
%%\end{itemize}
%%\end{frame}
%%
%%\hypertarget{part-2-milestone-experiment}{%
%%\section{Part 2: MILESTONE
%%experiment}\label{part-2-milestone-experiment}}
%%
%%\hypertarget{motivation}{%
%%\subsection{Motivation}\label{motivation}}
%%
%%\begin{frame}{Motivation}
%%\begin{itemize}
%%\tightlist
%%\item
%%  Strongly stratified turbulence regime characterized by
%%  \textbf{horizontal Froude number} and \textbf{buoyancy Reynolds
%%  number}: \[
%%  F_h = \frac{\epsK}{NU^2} \ll 1,\text{ and } \R = ReF_h^2 = \frac{\epsK}{\nu N^2} > 10.
%%  \]
%%\end{itemize}
%%
%%\pause
%%
%%\begin{itemize}
%%\tightlist
%%\item
%%  \textbf{Mixing efficiency} (\(\eta\)) or \textbf{mixing coefficient}
%%  (\(\Gamma\)) are defined as: \[
%%  \eta = \epsP / (\epsK + \epsP),\quad \Gamma = \epsP / \epsK,
%%  \] and widely used to parametrize ocean eddy-diffusivity turbulence
%%  models.
%%\end{itemize}
%%
%%\pause
%%
%%\begin{columns}[T]
%%\begin{column}{0.48\textwidth}
%%\begin{figure}
%%\centering
%%\includegraphics{../imgs/exp-zigzag.png}
%%\caption{Visuals of ``zig-zag instability'' from Billant \&
%%Chomaz(2000)}
%%\end{figure}
%%\end{column}
%%
%%\begin{column}{0.5\textwidth}
%%\begin{block}{Verify results from stratified turbulence theories}
%%\protect\hypertarget{verify-results-from-stratified-turbulence-theories}{}
%%\begin{enumerate}
%%\tightlist
%%\item
%%  \textbf{layered structures} with
%%  \(l_v \sim u/N\)\footnote<.->[frame]{\citet{billant_experimental_2000}}
%%\end{enumerate}
%%
%%~
%%
%%\pause
%%
%%\begin{enumerate}
%%\setcounter{enumi}{1}
%%\tightlist
%%\item
%%  \textbf{forward energy cascade} with
%%  spectrum\footnote<.->[frame]{\citet{Lindborg2006}}
%%  \(E(k) \sim \epsilon^{2/3} k^{-5/3}\), or \(2^{nd}\)-order structure
%%  function scaling\footnote<.->[frame]{\citet{ChoLindborg2001}} as
%%  \(\langle\delta \mathbf{u}. \delta \mathbf{u}\rangle \sim \epsilon^{2/3} r^{2/3}\),
%%  and
%%\end{enumerate}
%%
%%~
%%
%%\pause
%%
%%\begin{enumerate}
%%\setcounter{enumi}{2}
%%\tightlist
%%\item
%%  measure the \textbf{mixing efficiency} in the strongly stratified
%%  regime\footnote<.->[frame]{\citet{maffioli_mixing_2016}}.
%%\end{enumerate}
%%\end{block}
%%\end{column}
%%\end{columns}
%%\end{frame}
%%
%%\hypertarget{experimental-setup}{%
%%\subsection{Experimental Setup}\label{experimental-setup}}
%%
%%\begin{frame}{Experimental Setup}
%%\begin{columns}[T]
%%\begin{column}{0.5\textwidth}
%%\begin{figure}
%%\centering
%%
%%\subfloat[Schematic of the Coriolis platform and mounted instruments
%%(top
%%view)]{\includegraphics[width=\textwidth,height=0.4\textheight]{../paper_05_milestone_issf/Figures/scheme_exp_grid_MILESTONE_Euhit.pdf}\label{fig:scheme-coriolis}}
%%
%%\subfloat[Top view of the
%%setup]{\includegraphics[width=\textwidth,height=0.3\textheight]{../imgs/MILESTONE/GOPR1465.JPG}\label{fig:exp-top}}
%%
%%\caption{Experimental setup}
%%
%%\label{fig:scheme}
%%
%%\end{figure}
%%\end{column}
%%
%%\begin{column}{0.48\textwidth}
%%\begin{block}{Equipment}
%%\protect\hypertarget{equipment}{}
%%\begin{itemize}
%%\tightlist
%%\item
%%  1 \textbf{horizontal} scanning (2D-2C) PIV system
%%\item
%%  1 \textbf{vertical} stereoscopic (2D-3C) PIV system
%%\item
%%  5 conductometric / density \textbf{probes}
%%\item
%%  1 oscillating \textbf{carriage} with 6 cylinders of diameter 0.25 m
%%\item
%%  open-source software for control, calibration, data acquisition etc.
%%\end{itemize}
%%
%%~
%%
%%\centering{%
%%  {%
%%  \movie[
%%    width=7cm,
%%    height=4cm,
%%    showcontrols,
%%    poster,
%%    autostart, loop
%%  ]{}{./videos/moving_carriage.mp4}
%%
%%  Video: Carriage in the Coriolis platform}
%%}
%%\end{block}
%%\end{column}
%%\end{columns}
%%\end{frame}
%%
%%\hypertarget{preliminary-results}{%
%%\subsection{Preliminary results}\label{preliminary-results}}
%%
%%\begin{frame}{Preliminary results: layered
%%structures\footnote<.->[frame]{\citet{ISSF2016}}}
%%\protect\hypertarget{preliminary-results-layered-structures}{}
%%\begin{itemize}
%%\tightlist
%%\item
%%  Vertical length scale\footnote<.->[frame]{\citet{Billant2001}},
%%  \(l_v = u / N\)
%%\end{itemize}
%%
%%\begin{figure}
%%\centerline{
%%\includegraphics[height=0.39\textheight]{../paper_05_milestone_issf/Figures/exp21/vh_400.pdf}
%%\includegraphics[height=0.39\textheight]{../paper_05_milestone_issf/Figures/exp21/vh_655.pdf}
%%\includegraphics[height=0.39\textheight]{../paper_05_milestone_issf/Figures/exp21/vh_890.pdf}}
%%\vspace{0mm}
%%\centerline{
%%\includegraphics[height=0.37\textheight]{../paper_05_milestone_issf/Figures/exp21/vv_890.pdf}
%%\includegraphics[height=0.37\textheight]{../paper_05_milestone_issf/Figures/exp21/vv_655.pdf}
%%\includegraphics[height=0.37\textheight]{../paper_05_milestone_issf/Figures/exp21/vv_400.pdf}
%%}
%%\vspace{-2mm}
%%\caption{Instantaneous horizontal (top, $z=40$~cm) and vertical
%%fields (bottom) for $F_{hc} = 0.1$ and $\mathcal{R}_c=450$.}
%%\label{fig:field}
%%\end{figure}
%%\vspace{-2mm}
%%\end{frame}
%%
%%\begin{frame}{Preliminary results: horizontal structure
%%function\footnote<.->[frame]{\citet{ISSF2016}}}
%%\protect\hypertarget{preliminary-results-horizontal-structure-function}{}
%%\begin{itemize}
%%\tightlist
%%\item
%%  Second-order structure
%%  function\footnote<.->[frame]{\citet{ChoLindborg2001}}
%%  \(S_h \sim \epsilon^{2/3} r^{2/3}\)
%%\end{itemize}
%%
%%\begin{figure}[ht!]
%%\centerline{
%%\includegraphics[height=0.7\textheight]{../paper_05_milestone_issf/Figures/exp28/normalized_S2_exp28.pdf}}
%%\caption{Normalized $S_h$  as a function of
%%$r/M$ for $F_{hc} = 0.1$ and $\mathcal{R}_c=450$.}
%%\label{fig:S2}
%%\end{figure}
%%\end{frame}
%%
%%\begin{frame}{Preliminary results:
%%mixing\footnote<.->[frame]{\citet{ISSF2016}}}
%%\protect\hypertarget{preliminary-results-mixing}{}
%%\begin{itemize}
%%\tightlist
%%\item
%%  \textbf{Mixing
%%  coefficient}\footnote<.->[frame]{\citet{maffioli_mixing_2016}} in
%%  strongly stratified regime \(\Gamma \to 0.2 ?\) and in weakly
%%  stratified regime \(\Gamma \sim F_h^{-2}\).
%%\end{itemize}
%%
%%\begin{figure}[hb!]
%%\centerline{
%%\includegraphics[width=0.45\textwidth]{../_paper_06_milestone/1st/tmp/fig_energy_pot_vs_time}
%%\includegraphics[width=0.55\textwidth]{../_paper_06_milestone/1st/tmp/fig_dt_pot_energy}
%%}
%%\caption{Evolution of potential energy $E_P$ normalized by linear stratification for
%%experiment M17-21 (left) and normalized mixing coefficient $\eps_P /
%%(3\times10^{-3} {U_c}^3/D_c)$ for some MILESTONE 17 experiments (right).}%
%%\label{fig:dt:pot:energy}
%%\end{figure}
%%\end{frame}
%%
%%\hypertarget{part-3-reproducible-open-science-through-open-source}{%
%%\section{Part 3: Reproducible open science through open
%%source}\label{part-3-reproducible-open-science-through-open-source}}
%%
%%\hypertarget{open-science}{%
%%\subsection{Open science}\label{open-science}}
%%
%%\begin{frame}{Open science}
%%\begin{columns}[T]
%%\begin{column}{0.48\textwidth}
%%\includegraphics[width=0.9\textwidth,height=\textheight]{../imgs/open_science.pdf}
%%\end{column}
%%
%%\begin{column}{0.6\textwidth}
%%\begin{block}{Path to reproducible research}
%%\protect\hypertarget{path-to-reproducible-research}{}
%%\pause
%%
%%\begin{itemize}
%%\tightlist
%%\item
%%  Accessible knowledge: {\color{violet}{open access}}
%%\end{itemize}
%%
%%\pause
%%
%%\begin{itemize}
%%\item
%%  Accessible implementation:
%%  {\color{Cerulean}{{license + open source code}}}
%%\item
%%  Reliability: {\color{Cerulean}{documentation, continuous integration}}
%%\end{itemize}
%%
%%\pause
%%
%%\begin{itemize}
%%\item
%%  Open data: {\color{SeaGreen}{citable datasets}}
%%\item
%%  Tracking workflow: {\color{SeaGreen}{version control}}
%%\end{itemize}
%%
%%\pause
%%
%%\begin{itemize}
%%\tightlist
%%\item
%%  Publish: {\color{DarkGoldenrod}{manuscript + data + code + workflow}}
%%\end{itemize}
%%
%%\pause
%%\end{block}
%%
%%\begin{block}{Arguments for open source}
%%\protect\hypertarget{arguments-for-open-source}{}
%%\begin{itemize}
%%\item
%%  Reproducibility
%%\item
%%  Peer review for both manuscript and code
%%\item
%%  Interoperable and sustainable
%%\item
%%  Public money, public code\footnote<.->[frame]{https://publiccode.eu}
%%\end{itemize}
%%
%%\pause
%%\end{block}
%%
%%\begin{block}{Arguments against open source}
%%\protect\hypertarget{arguments-against-open-source}{}
%%\begin{itemize}
%%\item
%%  Comparative advantage
%%\item
%%  Lack of support
%%\item
%%  Curb industrial usage
%%\item
%%  Lack of documentation / not near production quality
%%\end{itemize}
%%\end{block}
%%\end{column}
%%\end{columns}
%%\end{frame}
%%
%%\begin{frame}{Python programming language}
%%\protect\hypertarget{python-programming-language}{}
%%\begin{itemize}
%%\item
%%  One of the most popular
%%  languages\footnote<.->[frame]{Stack Overflow, GitHub, TIOBE index, IEEE spectrum}
%%  in the world
%%\item
%%  General purpose
%%\item
%%  Thriving scientific community
%%\end{itemize}
%%
%%\begin{columns}[T]
%%\begin{column}{0.5\textwidth}
%%\includegraphics[width=0.9\textwidth,height=\textheight]{../imgs/python-sci-ecosystem.png}
%%\end{column}
%%
%%\begin{column}{0.48\textwidth}
%%\includegraphics[width=0.9\textwidth,height=\textheight]{../imgs/python-popularity.png}
%%\end{column}
%%\end{columns}
%%\end{frame}
%%
%%\begin{frame}{Why Python in sciences?}
%%\protect\hypertarget{why-python-in-sciences}{}
%%\begin{greencolorbox}[Advantages]\label{advantages}
%%
%%\begin{itemize}
%%\item
%%  Simple learning curve
%%\item
%%  Rapid prototyping
%%\item
%%  Expressive to communicate ideas
%%\item
%%  Extensible
%%\item
%%  Batteries included: powerful standard libraries
%%\end{itemize}
%%
%%\end{greencolorbox}
%%
%%\begin{bluecolorbox}[Issues and solutions]\label{issues-and-solutions}
%%
%%\begin{itemize}
%%\tightlist
%%\item
%%  CPU bounded performance
%%
%%  \begin{itemize}
%%  \tightlist
%%  \item
%%    native (C, C++ or Fortran), compiled (AOT or JIT)
%%    \textbf{extensions} for hotspots
%%  \end{itemize}
%%\item
%%  Concurrent, but no parallel threading
%%
%%  \begin{itemize}
%%  \tightlist
%%  \item
%%    use \textbf{multiprocessing} / \textbf{MPI}
%%  \end{itemize}
%%\end{itemize}
%%
%%\end{bluecolorbox}
%%\end{frame}
%%
%%\hypertarget{fluiddyn-project}{%
%%\subsection{FluidDyn project}\label{fluiddyn-project}}
%%
%%\begin{frame}[fragile]{FluidDyn project}
%%\begin{columns}[T]
%%\begin{column}{0.45\textwidth}
%%\begin{figure}
%%\centering
%%\includegraphics[width=0.9\textwidth,height=\textheight]{../imgs/logo-fluiddyn.jpg}
%%\caption{Project to foster open-science and open-source in fluid
%%mechanics}
%%\end{figure}
%%
%%\pause
%%
%%\begin{itemize}
%%\item
%%  {\alert<+->{\texttt{fluiddyn}}}: base
%%  package\footnote<.->[frame]{\citet{fluiddyn}}
%%\item
%%  {\alert<+->{\texttt{fluidfft}}}: API for Fast Fourier Transforms
%%\item
%%  {\alert<+->{\texttt{fluidsim}}}: CFD framework
%%\end{itemize}
%%
%%\onslide<+->
%%
%%\begin{itemize}
%%\item
%%  \texttt{fluidimage}: asynchronously parallelized image processing,
%%  including PIV
%%\item
%%  \texttt{fluidlab}: laboratory experiments
%%\item
%%  \texttt{transonic}: front-end for generating Python extensions
%%\end{itemize}
%%\end{column}
%%
%%\begin{column}{0.48\textwidth}
%%\begin{figure}
%%\centering
%%\includegraphics[width=0.9\textwidth,height=\textheight]{../imgs/dependency.pdf}
%%\caption{Standing on the shoulders of giants}
%%\end{figure}
%%\end{column}
%%\end{columns}
%%\end{frame}
%%
%%\begin{frame}[fragile]{Package
%%\texttt{fluidfft}\footnote<.->[frame]{\citet{fluidfft}}}
%%\protect\hypertarget{package-fluidfft}{}
%%\begin{columns}[T]
%%\begin{column}{0.4\textwidth}
%%\begin{itemize}
%%\item
%%  FFT libraries: \texttt{FFTW}, \texttt{P3DFFT}, \texttt{PFFT},
%%  \texttt{cuFFT} interfaced using \texttt{C++} and \texttt{Cython}
%%\item
%%  Pseudospectral ``operator'' classes with \texttt{Pythran} methods
%%\end{itemize}
%%\end{column}
%%
%%\begin{column}{0.48\textwidth}
%%\begin{Shaded}
%%\begin{Highlighting}[]
%%\CommentTok{\# An example for calculating gradient}
%%\ImportTok{from}\NormalTok{ fluidfft.fft2d.operators }\ImportTok{import}\NormalTok{ OperatorsPseudoSpectral2D}
%%\ImportTok{from}\NormalTok{ numpy }\ImportTok{import}\NormalTok{ sin, pi}
%%
%%\NormalTok{oper }\OperatorTok{=}\NormalTok{ OperatorsPseudoSpectral2D(}
%%\NormalTok{  nx}\OperatorTok{=}\DecValTok{100}\NormalTok{, ny}\OperatorTok{=}\DecValTok{100}\NormalTok{, lx}\OperatorTok{=}\DecValTok{2}\OperatorTok{*}\NormalTok{pi, ly}\OperatorTok{=}\DecValTok{2}\OperatorTok{*}\NormalTok{pi, fft}\OperatorTok{=}\StringTok{"fft2d.with\_fftw2d"}
%%\NormalTok{)}
%%\NormalTok{u }\OperatorTok{=}\NormalTok{ sin(oper.XX }\OperatorTok{+}\NormalTok{ oper.YY)}
%%\NormalTok{u\_fft }\OperatorTok{=}\NormalTok{ oper.fft(u)}
%%\NormalTok{px\_u\_fft, py\_u\_fft }\OperatorTok{=}\NormalTok{ oper.gradfft\_from\_fft(u\_fft)}
%%\end{Highlighting}
%%\end{Shaded}
%%\end{column}
%%\end{columns}
%%
%%\begin{figure}
%%\centering
%%\includegraphics[width=\textwidth,height=0.5\textheight]{../paper_01_fluidfft/Pyfig/fig_classes.pdf}
%%\caption{Class hierarchy of {\color{Salmon}{\emph{sequential}}},
%%{\color{magenta}{\emph{CUDA}}}, {\color{SeaGreen}{\emph{MPI}}} FFT
%%libraries}
%%\end{figure}
%%\end{frame}
%%
%%\begin{frame}{Performance of
%%\texttt{fluidfft}\footnote<.->[frame]{\citet{fluidfft}}: scaling}
%%\protect\hypertarget{performance-of-fluidfft-scaling}{}
%%\begin{figure}
%%\centering
%%\includegraphics[width=1.3\textwidth,height=\textheight]{../paper_01_fluidfft/tmp/fig_beskow_1152x1152x1152.pdf}
%%\caption{Strong scaling of 3D FFT upto 10000 cores}
%%\end{figure}
%%\end{frame}
%%
%%\begin{frame}[fragile]{Performance of
%%\texttt{fluidfft}\footnote<.->[frame]{\citet{fluidfft}}:
%%microbenchmarks}
%%\protect\hypertarget{performance-of-fluidfft-microbenchmarks}{}
%%\begin{itemize}
%%\tightlist
%%\item
%%  \texttt{Pythran} extensions comparable compiled native languages
%%\item
%%  Memory allocation is expensive: \textbf{in-place} better than
%%  \textbf{out-of-place}
%%\end{itemize}
%%
%%\begin{figure}
%%\centering
%%\includegraphics[width=1\textwidth,height=\textheight]{../paper_01_fluidfft/tmp/fig_microbench.pdf}
%%\caption{Microbenchmarks of projection function}
%%\end{figure}
%%\end{frame}
%%
%%\begin{frame}[fragile]{Package
%%\texttt{fluidsim}\footnote<.->[frame]{\citet{fluidsim}}}
%%\protect\hypertarget{package-fluidsim}{}
%%\begin{columns}[T]
%%\begin{column}{0.4\textwidth}
%%\begin{itemize}
%%\item
%%  Extensible, object-oriented CFD framework
%%\item
%%  On-the-fly post-processing
%%\item
%%  Code reuse: \texttt{numpy}, \texttt{fluiddyn}, \texttt{fluidfft} etc.
%%\end{itemize}
%%\end{column}
%%
%%\begin{column}{0.48\textwidth}
%%\begin{Shaded}
%%\begin{Highlighting}[]
%%\CommentTok{\# An example for running 3D Navier Stokes solver}
%%\ImportTok{from}\NormalTok{ fluidsim.solvers.ns3d.solver }\ImportTok{import}\NormalTok{ Simul}
%%
%%\NormalTok{params }\OperatorTok{=}\NormalTok{ Simul.create\_default\_params()}
%%\CommentTok{\# Modify parameters as needed}
%%\NormalTok{sim }\OperatorTok{=}\NormalTok{ Simul(params)}
%%\NormalTok{sim.time\_stepping.start()}
%%\end{Highlighting}
%%\end{Shaded}
%%\end{column}
%%\end{columns}
%%
%%\pause
%%
%%\begin{figure}
%%\centering
%%\includegraphics[width=\textwidth,height=0.55\textheight]{../paper_02_fluidsim/tmp/fig_profile3d.pdf}
%%\caption{Profiling \texttt{fluidsim.solvers.ns3d.solver}}
%%\end{figure}
%%\end{frame}
%%
%%\begin{frame}[fragile]{Performance of
%%\texttt{fluidsim}\footnote<.->[frame]{\citet{fluidsim}}: profiling}
%%\protect\hypertarget{performance-of-fluidsim-profiling}{}
%%\begin{figure}
%%\centering
%%\includegraphics[width=1.3\textwidth,height=\textheight]{../paper_02_fluidsim/tmp/fig_compare_with_ns3d.pdf}
%%\caption{Comparison of {\color{blue}{Fortran code NS3D}} with
%%{\color{DarkGoldenrod}{\texttt{fluidsim.solvers.ns3d.solver}}}}
%%\end{figure}
%%\end{frame}
%%
%%\begin{frame}[fragile]{Performance of
%%\texttt{fluidsim}\footnote<.->[frame]{\citet{fluidsim}}: scaling}
%%\protect\hypertarget{performance-of-fluidsim-scaling}{}
%%\begin{figure}
%%\centering
%%\includegraphics[width=1.3\textwidth,height=\textheight]{../paper_02_fluidsim/tmp/fig_bench_strong3d.pdf}
%%\caption{Strong scaling for \texttt{fluidsim.solvers.ns3d.solver} using
%%{\color{MidnightBlue}{FFTW-MPI}} and {\color{BurntOrange}{P3DFFT}} via
%%\texttt{fluidfft}}
%%\end{figure}
%%\end{frame}
%%
%%\hypertarget{concluding-remarks}{%
%%\section{Concluding remarks}\label{concluding-remarks}}
%%
%%\begin{frame}{Concluding remarks}
%%\protect\hypertarget{concluding-remarks-1}{}
%%\begin{block}{Part 1}
%%\protect\hypertarget{part-1}{}
%%\begin{itemize}
%%\item
%%  Scaling relations for \textbf{shock} dominated turbulence
%%\item
%%  Developed a toy-model which exhibits \textbf{forward cascade} in
%%  ``mesoscales'' with a \(k^{-5/3}\) spectrum
%%\end{itemize}
%%
%%\pause
%%\end{block}
%%
%%\begin{block}{Part 2}
%%\protect\hypertarget{part-2}{}
%%\begin{itemize}
%%\item
%%  Verified \textbf{vertical length scale} \(l_v \sim u/N\) and
%%  \textbf{forward cascade} \(S_h \sim \epsilon^{2/3}r^{2/3}\)
%%\item
%%  Mixing coefficient (\(\Gamma\)) estimates requires further
%%  investigations
%%\end{itemize}
%%
%%\pause
%%\end{block}
%%
%%\begin{block}{Part 3}
%%\protect\hypertarget{part-3}{}
%%\begin{itemize}
%%\item
%%  Python as primary language for building research software
%%\item
%%  Good \textbf{performance} by optimising bottlenecks
%%\end{itemize}
%%\end{block}
%%\end{frame}
%%
