%------------------------------------------------------------------------------
% Define title, author(s), affiliation and publishing status
%
\papertitle[Frequency dependence of the complex viscosity of a dense noncolloidal suspension] % Short title used in healines (optional)
{%
 Anomalous frequency dependence of the complex viscosity of a dense noncolloidal particle suspension% THE COMMENT SYMBOL AT THE END OF THIS LINE IS NEEDED
}%
%
\papertoctitle{Anomalous frequency dependence of the complex viscosity of a dense noncolloidal particle suspension} % Title for toc
%
\paperauthor[Ge, Martone, Carotenuto, Radhakrishnan, Brandt, Minale] % Short authors used in headlines and List Of Papers
{%
  Zhouyang Ge$^1$, Raffaella Martone$^2$, Claudia Carotenuto$^2$, Rangarajan Radhakrishnan$^3$, Luca Brandt$^1$, Mario Minale$^2$%
}%
%
\listpaperauthor{Z. Ge, R. Martone, C. Carotenuto, R. Radhakrishnan, L. Brandt, M. Minale}% (optional) Short authors used in List Of Papers
%
\paperaffiliation
{%
  $^1$ Linn\'e FLOW Centre and SeRC, KTH Mechanics, S-100 44 Stockholm, Sweden\\%
  $^2$ University of Campania ``Luigi Vanvitelli'', Department of Engineering, \\via Roma 29 -- 81031 Aversa (CE), Italy\\%
  $^3$ Institute for Infrastructure and Environment, School of Engineering, \\The University of Edinburgh, United Kingdom%
}%
%
\paperjournal[To be submitted] % Short publish info used in List Of Papers
{%
	To be submitted%
}%
%
\papervolume{}%
%
\papernumber{}
%
\paperpages{}%
%
\paperyear{}%
%
\papersummary%
{% Insert summary of the paper here (used in introduction)
   In this letter, we reported rheological measurements of a noncolloidal particle suspension at 40\% solid volume fraction, displaying a frequency-dependent complex viscosity in oscillatory shear (OS) flows but a constant dynamic viscosity under the same shear rates in steady shear (SS) flows. Using the discrete element method developed in \emph{Paper 7}, we showed that this contradiction arises from the underlying microstructural difference between OS and SS, manifested by the stress budget difference and the suspension fabric statistics, and further predicted shear thickening or thinning, only in OS, due to repulsive or attractive interactions, respectively.
}%
%
\graphicspath{{paper8/}}%
%
%
%===============================================================================
%                            BEGIN PAPER
%===============================================================================
%
\begin{paper}

\makepapertitle

%------------------------------------------------------------------------------
% Abstract
%------------------------------------------------------------------------------
%
\begin{paperabstract}
	We report rheological measurements of a noncolloidal particle suspension at 40\% solid volume fraction. A frequency-dependent complex viscosity is found under oscillatory shear (OS) flows, whereas a constant dynamic viscosity is found under the same shear rates in steady shear (SS) flows. We hypothesize this contradiction arises from the underlying microstructural difference between OS and SS, mediated by interparticle forces. Discrete element simulations of a proxy noncolloidal suspension further reveals the qualitative difference of the stress budget and predicts shear thickening or thinning, only in OS, due to repulsive or attractive interactions, respectively.
\end{paperabstract}


%------------------------------------------------------------------------------
% Article
%------------------------------------------------------------------------------
%
%\documentclass[12pt, twocolumn]{article}
%\usepackage{helvet}
%
%\usepackage{epsfig}
%\usepackage[latin1]{inputenc}
%\begin{document}
%
%\title{Emulating Von Neumann Machines and Massive Multiplayer Online Role-
%Playing Games}
%\author{Mickey Mouse, Goofy G. Goof and Donald Duck}
%
%\date{}

%\maketitle




%\section*{Abstract}
%
% Many computational biologists would agree that, had it not been for
% Byzantine fault tolerance, the synthesis of replication that made
% developing and possibly investigating erasure coding a reality might
% never have occurred. In this work, we prove  the synthesis of linked
% lists. Even though such a hypothesis at first glance seems
% counterintuitive, it always conflicts with the need to provide
% object-oriented languages to systems engineers. APER, our new framework
% for mobile archetypes, is the solution to all of these grand
% challenges.



In \cite{Batchelor} we believe.



%\begin{footnotesize}
%\bibliography{scigenbibfile.Donald+Duck.Mickey+Mouse.Goofy+G.+Goof}\bibliographystyle{acm}
%\end{footnotesize}
%
%\end{document}



%------------------------------------------------------------------------------
% Bibliography
%------------------------------------------------------------------------------
%
%\clearpage
\bibliographystyle{jfm}
\bibliography{thesis}
%
\IfFileExists{paper8/paper.bbl}{%------------------------------------------------------------------------------
% Define title, author(s), affiliation and publishing status
%
\papertitle[Frequency dependence of the complex viscosity of a dense noncolloidal suspension] % Short title used in healines (optional)
{%
 Anomalous frequency dependence of the complex viscosity of a dense noncolloidal particle suspension% THE COMMENT SYMBOL AT THE END OF THIS LINE IS NEEDED
}%
%
\papertoctitle{Anomalous frequency dependence of the complex viscosity of a dense noncolloidal particle suspension} % Title for toc
%
\paperauthor[Ge, Martone, Carotenuto, Brandt, Minale] % Short authors used in headlines and List Of Papers
{%
  Zhouyang Ge$^1$, Raffaella Martone$^2$, Claudia Carotenuto$^2$, Luca Brandt$^1$, Mario Minale$^2$%
}%
%
\listpaperauthor{Z. Ge, R. Martone, C. Carotenuto, L. Brandt, M. Minale}% (optional) Short authors used in List Of Papers
%
\paperaffiliation
{%
  $^1$ Linn\'e FLOW Centre and SeRC, KTH Mechanics, S-100 44 Stockholm, Sweden\\%
  $^2$ University of Campania ``Luigi Vanvitelli'', Department of Engineering, \\via Roma 29 -- 81031 Aversa (CE), Italy%
}%
%
\paperjournal[To be submitted] % Short publish info used in List Of Papers
{%
	To be submitted%
}%
%
\papervolume{}%
%
\papernumber{}
%
\paperpages{}%
%
\paperyear{}%
%
\papersummary%
{% Insert summary of the paper here (used in introduction)
   In this letter, we reported rheological measurements of a noncolloidal particle suspension at 40\% solid volume fraction, displaying a frequency-dependent complex viscosity in oscillatory shear (OS) flows but a constant dynamic viscosity under the same shear rates in steady shear (SS) flows. Using the discrete element method developed in \emph{Paper 7}, we showed that this contradiction arises from the underlying microstructural difference between OS and SS, manifested by the stress budget difference and the suspension fabric statistics, and further predicted shear thickening or thinning, only in OS, due to repulsive or attractive interactions, respectively.
}%
%
\graphicspath{{paper8/}}%
%
%
%===============================================================================
%                            BEGIN PAPER
%===============================================================================
%
\begin{paper}

\makepapertitle

%------------------------------------------------------------------------------
% Abstract
%------------------------------------------------------------------------------
%
\begin{paperabstract}
	We report rheological measurements of a noncolloidal particle suspension at 40\% solid volume fraction. A frequency-dependent complex viscosity is found under oscillatory shear (OS) flows, whereas a constant dynamic viscosity is found under the same shear rates in steady shear (SS) flows. We hypothesize this contradiction arises from the underlying microstructural difference between OS and SS, mediated by interparticle forces. Discrete element simulations of a proxy noncolloidal suspension further reveals the qualitative difference of the stress budget and predicts shear thickening or thinning, only in OS, due to repulsive or attractive interactions, respectively.
\end{paperabstract}


%------------------------------------------------------------------------------
% Article
%------------------------------------------------------------------------------
%
%\documentclass[12pt, twocolumn]{article}
%\usepackage{helvet}
%
%\usepackage{epsfig}
%\usepackage[latin1]{inputenc}
%\begin{document}
%
%\title{Emulating Von Neumann Machines and Massive Multiplayer Online Role-
%Playing Games}
%\author{Mickey Mouse, Goofy G. Goof and Donald Duck}
%
%\date{}

%\maketitle




%\section*{Abstract}
%
% Many computational biologists would agree that, had it not been for
% Byzantine fault tolerance, the synthesis of replication that made
% developing and possibly investigating erasure coding a reality might
% never have occurred. In this work, we prove  the synthesis of linked
% lists. Even though such a hypothesis at first glance seems
% counterintuitive, it always conflicts with the need to provide
% object-oriented languages to systems engineers. APER, our new framework
% for mobile archetypes, is the solution to all of these grand
% challenges.



In \cite{Batchelor} we believe.



%\begin{footnotesize}
%\bibliography{scigenbibfile.Donald+Duck.Mickey+Mouse.Goofy+G.+Goof}\bibliographystyle{acm}
%\end{footnotesize}
%
%\end{document}



%------------------------------------------------------------------------------
% Bibliography
%------------------------------------------------------------------------------
%
%\clearpage
\bibliographystyle{jfm}
\bibliography{thesis}
%
\IfFileExists{paper8/paper.bbl}{%------------------------------------------------------------------------------
% Define title, author(s), affiliation and publishing status
%
\papertitle[Frequency dependence of the complex viscosity of a dense noncolloidal suspension] % Short title used in healines (optional)
{%
 Anomalous frequency dependence of the complex viscosity of a dense noncolloidal particle suspension% THE COMMENT SYMBOL AT THE END OF THIS LINE IS NEEDED
}%
%
\papertoctitle{Anomalous frequency dependence of the complex viscosity of a dense noncolloidal particle suspension} % Title for toc
%
\paperauthor[Ge, Martone, Carotenuto, Brandt, Minale] % Short authors used in headlines and List Of Papers
{%
  Zhouyang Ge$^1$, Raffaella Martone$^2$, Claudia Carotenuto$^2$, Luca Brandt$^1$, Mario Minale$^2$%
}%
%
\listpaperauthor{Z. Ge, R. Martone, C. Carotenuto, L. Brandt, M. Minale}% (optional) Short authors used in List Of Papers
%
\paperaffiliation
{%
  $^1$ Linn\'e FLOW Centre and SeRC, KTH Mechanics, S-100 44 Stockholm, Sweden\\%
  $^2$ University of Campania ``Luigi Vanvitelli'', Department of Engineering, \\via Roma 29 -- 81031 Aversa (CE), Italy%
}%
%
\paperjournal[To be submitted] % Short publish info used in List Of Papers
{%
	To be submitted%
}%
%
\papervolume{}%
%
\papernumber{}
%
\paperpages{}%
%
\paperyear{}%
%
\papersummary%
{% Insert summary of the paper here (used in introduction)
   In this letter, we reported rheological measurements of a noncolloidal particle suspension at 40\% solid volume fraction, displaying a frequency-dependent complex viscosity in oscillatory shear (OS) flows but a constant dynamic viscosity under the same shear rates in steady shear (SS) flows. Using the discrete element method developed in \emph{Paper 7}, we showed that this contradiction arises from the underlying microstructural difference between OS and SS, manifested by the stress budget difference and the suspension fabric statistics, and further predicted shear thickening or thinning, only in OS, due to repulsive or attractive interactions, respectively.
}%
%
\graphicspath{{paper8/}}%
%
%
%===============================================================================
%                            BEGIN PAPER
%===============================================================================
%
\begin{paper}

\makepapertitle

%------------------------------------------------------------------------------
% Abstract
%------------------------------------------------------------------------------
%
\begin{paperabstract}
	We report rheological measurements of a noncolloidal particle suspension at 40\% solid volume fraction. A frequency-dependent complex viscosity is found under oscillatory shear (OS) flows, whereas a constant dynamic viscosity is found under the same shear rates in steady shear (SS) flows. We hypothesize this contradiction arises from the underlying microstructural difference between OS and SS, mediated by interparticle forces. Discrete element simulations of a proxy noncolloidal suspension further reveals the qualitative difference of the stress budget and predicts shear thickening or thinning, only in OS, due to repulsive or attractive interactions, respectively.
\end{paperabstract}


%------------------------------------------------------------------------------
% Article
%------------------------------------------------------------------------------
%
%\documentclass[12pt, twocolumn]{article}
%\usepackage{helvet}
%
%\usepackage{epsfig}
%\usepackage[latin1]{inputenc}
%\begin{document}
%
%\title{Emulating Von Neumann Machines and Massive Multiplayer Online Role-
%Playing Games}
%\author{Mickey Mouse, Goofy G. Goof and Donald Duck}
%
%\date{}

%\maketitle




%\section*{Abstract}
%
% Many computational biologists would agree that, had it not been for
% Byzantine fault tolerance, the synthesis of replication that made
% developing and possibly investigating erasure coding a reality might
% never have occurred. In this work, we prove  the synthesis of linked
% lists. Even though such a hypothesis at first glance seems
% counterintuitive, it always conflicts with the need to provide
% object-oriented languages to systems engineers. APER, our new framework
% for mobile archetypes, is the solution to all of these grand
% challenges.



In \cite{Batchelor} we believe.



%\begin{footnotesize}
%\bibliography{scigenbibfile.Donald+Duck.Mickey+Mouse.Goofy+G.+Goof}\bibliographystyle{acm}
%\end{footnotesize}
%
%\end{document}



%------------------------------------------------------------------------------
% Bibliography
%------------------------------------------------------------------------------
%
%\clearpage
\bibliographystyle{jfm}
\bibliography{thesis}
%
\IfFileExists{paper8/paper.bbl}{%------------------------------------------------------------------------------
% Define title, author(s), affiliation and publishing status
%
\papertitle[Frequency dependence of the complex viscosity of a dense noncolloidal suspension] % Short title used in healines (optional)
{%
 Anomalous frequency dependence of the complex viscosity of a dense noncolloidal particle suspension% THE COMMENT SYMBOL AT THE END OF THIS LINE IS NEEDED
}%
%
\papertoctitle{Anomalous frequency dependence of the complex viscosity of a dense noncolloidal particle suspension} % Title for toc
%
\paperauthor[Ge, Martone, Carotenuto, Brandt, Minale] % Short authors used in headlines and List Of Papers
{%
  Zhouyang Ge$^1$, Raffaella Martone$^2$, Claudia Carotenuto$^2$, Luca Brandt$^1$, Mario Minale$^2$%
}%
%
\listpaperauthor{Z. Ge, R. Martone, C. Carotenuto, L. Brandt, M. Minale}% (optional) Short authors used in List Of Papers
%
\paperaffiliation
{%
  $^1$ Linn\'e FLOW Centre and SeRC, KTH Mechanics, S-100 44 Stockholm, Sweden\\%
  $^2$ University of Campania ``Luigi Vanvitelli'', Department of Engineering, \\via Roma 29 -- 81031 Aversa (CE), Italy%
}%
%
\paperjournal[To be submitted] % Short publish info used in List Of Papers
{%
	To be submitted%
}%
%
\papervolume{}%
%
\papernumber{}
%
\paperpages{}%
%
\paperyear{}%
%
\papersummary%
{% Insert summary of the paper here (used in introduction)
   In this letter, we reported rheological measurements of a noncolloidal particle suspension at 40\% solid volume fraction, displaying a frequency-dependent complex viscosity in oscillatory shear (OS) flows but a constant dynamic viscosity under the same shear rates in steady shear (SS) flows. Using the discrete element method developed in \emph{Paper 7}, we showed that this contradiction arises from the underlying microstructural difference between OS and SS, manifested by the stress budget difference and the suspension fabric statistics, and further predicted shear thickening or thinning, only in OS, due to repulsive or attractive interactions, respectively.
}%
%
\graphicspath{{paper8/}}%
%
%
%===============================================================================
%                            BEGIN PAPER
%===============================================================================
%
\begin{paper}

\makepapertitle

%------------------------------------------------------------------------------
% Abstract
%------------------------------------------------------------------------------
%
\begin{paperabstract}
	We report rheological measurements of a noncolloidal particle suspension at 40\% solid volume fraction. A frequency-dependent complex viscosity is found under oscillatory shear (OS) flows, whereas a constant dynamic viscosity is found under the same shear rates in steady shear (SS) flows. We hypothesize this contradiction arises from the underlying microstructural difference between OS and SS, mediated by interparticle forces. Discrete element simulations of a proxy noncolloidal suspension further reveals the qualitative difference of the stress budget and predicts shear thickening or thinning, only in OS, due to repulsive or attractive interactions, respectively.
\end{paperabstract}


%------------------------------------------------------------------------------
% Article
%------------------------------------------------------------------------------
%
\input{paper8/article.tex}


%------------------------------------------------------------------------------
% Bibliography
%------------------------------------------------------------------------------
%
%\clearpage
\bibliographystyle{jfm}
\bibliography{thesis}
%
\IfFileExists{paper8/paper.bbl}{\input{paper8/paper.bbl}}{}


%===============================================================================
%                            END PAPER
%===============================================================================
\end{paper}
}{}


%===============================================================================
%                            END PAPER
%===============================================================================
\end{paper}
}{}


%===============================================================================
%                            END PAPER
%===============================================================================
\end{paper}
}{}


%===============================================================================
%                            END PAPER
%===============================================================================
\end{paper}
